%%%%%%%%%%%%%%%%%%%%%%%%%%%%%%%%%%%%%%%%%%%%%%%%%%%%
%%%             Metadata                         %%%
%%%%%%%%%%%%%%%%%%%%%%%%%%%%%%%%%%%%%%%%%%%%%%%%%%%%      

\title{\textbf{Grundkurs Linguistik}}

\subtitle{Phonologie\\Sprachgebildelautlehre}

\author[aMyP]{
	{\small Antonio Machicao y Priemer}
%	\\
%	{\footnotesize \url{http://www.linguistik.hu-berlin.de/staff/amyp}\\
%	\href{mailto:mapriema@hu-berlin.de}{mapriema@hu-berlin.de}}
}

\institute{Institut für deutsche Sprache und Linguistik}

%%%%%%%%%%%%%%%%%%%%%%%%%      
\date{ }
%\publishers{\textbf{6. linguistischer Methodenworkshop \\ Humboldt-Universität zu Berlin}}

%\hyphenation{nobreak}


%%%%%%%%%%%%%%%%%%%%%%%%%%%%%%%%%%%%%%%%%%%%%%%%%%%%
%%%             Preamble's End                   %%%
%%%%%%%%%%%%%%%%%%%%%%%%%%%%%%%%%%%%%%%%%%%%%%%%%%%%      


%%%%%%%%%%%%%%%%%%%%%%%%%      
\begin{frame}
  \HUtitle
\end{frame}

\frame{
\begin{multicols}{2}
	\frametitle{Inhaltsverzeichnis}\tableofcontents
	%[pausesections]
\end{multicols}
}

%%%%%%%%%%%%%%%%%%%%%%%%%%%%%%%%%%%%%%%%%%%%%%%%%%%%%%%
%%%%%%%%%%%%%%%%%%%%%%%%%%%%%%%%%%%%%%%%%%%%%%%%%%%%%%%

\nocite{Repp&Co12a} 
\nocite{Hall00a} 
\nocite{Luedeling09a} 
\nocite{Ramers08a}

%%%%%%%%%%%%%%%%%%%%%%%%%%%%%%%%%%%%%%%%%%%%%%%%%%%%%%%
%
\section{Einführung}
%
%\frame{
%\begin{multicols}{2}
%\frametitle{~}
%	\tableofcontents[currentsection]
%\end{multicols}
%}
%%%%%%%%%%%%%%%%%%%%%%%%%%%%%%%%%%%%%%%%%%%%%%%%%%%%%%%

\begin{frame}{Einführung}

\begin{itemize}
	\item Trennung von Phonetik und Phonologie: Ende der 1920er Jahre
	\item[]
	\item Strukturalistische Lehre der Prager Schule (vgl. \cite{Trubetzkoy89a})
	\item[]
	\item Unterscheidung auf allen Ebenen zwischen
	
	\begin{itemize}
		\item[]
		\item Sprachgebilde (zugrunde liegendes System \ras \textit{langue} -- später \textit{Kompetenz})
		\item[]
		\item[] und
		\item[]
		\item Sprechakt (tatsächliche Realisierung in einer Kommunikationssituation \textit{parole} -- später \textit{Performanz})
	\end{itemize}
	
\end{itemize}

\end{frame}



%%%%%%%%%%%%%%%%%%%%%%%%%%%%%%%%%%%%%%%%%%%%%%%%%%%%%%%

\begin{frame}
\frametitle{Einführung}

\begin{itemize}
	\item Phonetik: Untersuchung der materiellen Seite des Sprechens (Phone)
	\item[]
	\item Phonologie: Systematik der Laute \ras Materielle (messbare) Daten der Phonetik werden in abstrakterer Art und Weise \textbf{systematisiert}
	
	\begin{itemize}
		\item \textbf{Phoneminventar}: Bedeutungsunterscheidende Lauter einer Sprache 

		\ex. Im Dt. bedeutungsunterscheidend \textipa{[v]} und \textipa{[f]}: \textipa{[v\t{aI}n]} vs. \textipa{[f\t{aI}n]}
		
		\ex. Deutsch: 16 Vokale \& 20 Konsonanten
		
		\ex. Rotokas (Papua): 5 Vokale \& 6 Konsonanten
		
		\ex. Mittelwert: 23 Konsonanten \& 8 Vokale

		\item \textbf{Allophonie}: Vorkommen vs. Nicht-Vorkommen (bzw. Variation) von Lauten in bestimmten Kontexten

		\ex. Wann kommt der \gqq{Ich-Laut} und wann der \gqq{Ach-Laut} vor?

	\end{itemize}
	
\end{itemize}
		
\end{frame}



%%%%%%%%%%%%%%%%%%%%%%%%%%%%%%%%%%%%%%%%%%%%%%%%%%%%%%%

\begin{frame}
\frametitle{Einführung}

\begin{itemize}
	\item \textbf{Phonologische Distribution}: An welchen Stellen kann ein Laut oder eine Lautfolge auftreten

	\ex. \textipa{[St\;R]} am Wortanfang aber nicht am Wortende \textipa{[St\;R\t{aU}x]} vs. *\textipa{[\ldots aSt\;R]}

	\item Phoneminventar, phonologische Distribution und Allophonie werden in der \textbf{strukturalistischen Phonologie} untersucht
	\item[]
	\item \textbf{Strukturalistische} Phonologie $\rightarrow$ Beschreibung von sprachlichen Daten 

\end{itemize}

\end{frame}


%%%%%%%%%%%%%%%%%%%%%%%%%%%%%%%%%%%%%%%%%%%%%%%%%%%%%%%

\begin{frame}
\frametitle{Einführung}

\begin{itemize}
	\item \textbf{Phonologische Prozesse}: Welche Lautfolgen, die an der Oberfläche unterschiedlich klingen, werden durch die Sprachnutzer trotzdem als Varianten eines zugrunde liegenden Musters erkannt?

\ex. \textipa{[ga\;Rt@n]} vs. 
\textipa{[ga:d\textsyllabic{n}]}

	\item \textbf{Generative} Phonologie $\rightarrow$ Zugrundeliegende Form +  Regeln ($\rightarrow$ Schlüsse über die allgemeine Sprachfähigkeit!) 
	\item[]
	\item Aufgaben des phonologischen Moduls:
	
	\begin{itemize}
		\item Bildung (und Verständnis) wohlgeformter Lautketten
		\item Inventar von Minimaleinheiten (Distinktive Merkmale -- hier Phoneme!)
		\item Regelinventar
	\end{itemize}
	 
\end{itemize}

\end{frame}



%%%%%%%%%%%%%%%%%%%%%%%%%%%%%%%%%%%%%%%%%%%%%%%%%%%%%%%

\begin{frame}
\frametitle{Einführung}

\begin{itemize}
	\item Weitere Untersuchungsgebiete der Phonologie:
	
	\begin{itemize}
		\item[]
		\item Eigenschaften von (lautlichen) Einheiten, die größer sind als ein Laut (\zB \textbf{Silbenphonologie})
		\item[]
		\item Wortakzent (\textbf{metrische Phonologie})
		\item[]
		\item Satzakzent, Phrasierung, Pausen, Sprechmelodie (\textbf{prosodische Phonologie}, Intonation)
	\end{itemize}
	
	\item[]
	\item Betrachtung der Laute \ras \textbf{lineare Phonologie}
	\item[]
	\item Analyse von einer Silbe \ras \textbf{nicht lineare (hierarchische) Phonologie}
\end{itemize}

\end{frame}



%%%%%%%%%%%%%%%%%%%%%%%%%%%%%%%%%%%%%%%%%%%%%%%%%%%%%%%
%%%%%%%%%%%%%%%%%%%%%%%%%%%%%%%%%%%%%%%%%%%%%%%%%%%%%%%
%
\section{Phonem, Phon, Allophon}
\frame{
\begin{multicols}{2}
\frametitle{~}
	\tableofcontents[currentsection]
\end{multicols}
}
%%%%%%%%%%%%%%%%%%%%%%%%%%%%%%%%%%%%%%%%%%%%%%%%%%%%%%%

\begin{frame}{Phonem, Phon, Allophon}

\begin{itemize}
	\item \textbf{Phon} (Notation \textipa{[ ]}):
	
	\begin{itemize}
		\item[]
		\item Minimaleinheit der Phonetik
		\item Physikalisch messbare lautliche Einheit einer Sprache
	\end{itemize}
	
	\item[]
	\item \textbf{Phonem} (Notation \textipa{/ /}):
	
	\begin{itemize}
		\item[]
		\item Minimaleinheit der Phonologie
		\item Abstraktes Konstrukt, das für eine \textbf{Menge} von möglichen Phonen (Allophonen) steht
		\item Resultat von \textbf{Systematisierung}
		\item Ermittelbar durch \textbf{Minimalpaarbildung} (strukturalistisches Kriterium)
		
	\begin{block}{Minimalpaar}
Wortpaar, das sich nur in einem Laut (eher Phonem) an der gleichen Stelle unterscheidet.
	\end{block}
	
	\end{itemize}
	
\end{itemize}

\end{frame}



%%%%%%%%%%%%%%%%%%%%%%%%%%%%%%%%%%%%%%%%%%%%%%%%%%%%%%%

\begin{frame}
\frametitle{Phonem, Phon, Allophon}

\begin{itemize}
	\item \textbf{Phonem} (Notation \textipa{/ /}):
	
	\begin{itemize}
		\item Ermittelbar durch \textbf{Minimalpaarbildung} (strukturalistisches Kriterium)
		
		\begin{block}{Minimalpaar}
Wortpaar, das sich nur in einem Laut (eher Phonem) an der gleichen Stelle unterscheidet
		\end{block}
	
		\ex.\label{schaf} \textipa{[Sa:l]} $\langle$Schal$\rangle$ vs. \textipa{[Sa:f]} $\langle$Schaf$\rangle$
		
		\ex.\label{schall} \textipa{[Sa:l]} $\langle$Schal$\rangle$ vs. \textipa{[Sal]} $\langle$Schall$\rangle$
		
		\ex.\label{saal} \textipa{[Sa:l]} $\langle$Schal$\rangle$ vs. \textipa{[za:l]} $\langle$Saal$\rangle$
		
		\item \textbf{Phonologische Opposition}: Austausch der Laute wirkt sich bedeutungsunterscheidend (oder kategorieunterscheidend) aus.\\
		\textipa{/l/} vs. \textipa{/f/} in (\ref{schaf})\\
		\textipa{/a:/} vs. \textipa{/a/} in (\ref{schall})\\
		\textipa{/S/} vs. \textipa{/z/} in (\ref{saal})				
	\end{itemize}
	
\end{itemize}

\end{frame}



%%%%%%%%%%%%%%%%%%%%%%%%%%%%%%%%%%%%%%%%%%%%%%%%%%%%%%%

\begin{frame}
\frametitle{Phonem, Phon, Allophon}

\begin{block}{Phonem (strukturalistisch)}
Kleinste bedeutungsunterscheidende Einheit eines Sprachsystems
\end{block}

\begin{itemize}
	\item Ein Phonem trägt keine Bedeutung. Es unterscheidet Bedeutungen!
	\item[]
	\item Phoneme sind immer Phoneme \textbf{einer Sprache / eines Systems}

	\ex. Deutsch: \textipa{[papa]} = \textipa{[p\super{h}ap\super{h}a]}
	\ex. Hindi: \textipa{[pal]} ($\lsem$sich kümmern um$\rsem$) $\neq$ \textipa{[p\super{h}al]} ($\lsem$Messerblatt$\rsem$) 

\end{itemize}

\end{frame}



%%%%%%%%%%%%%%%%%%%%%%%%%%%%%%%%%%%%%%%%%%%%%%%%%%%%%%%

\begin{frame}
\frametitle{Phonem, Phon, Allophon}

\begin{itemize}
	\item \textbf{Allophone}:
	
	\begin{itemize}
		\item Phonetische Realisierungsvarianten \textbf{eines} Phonems
		
		\ex. \textipa{[Sp r a:xe]} = \textipa{[Sp \;R a:xe]} = \textipa{[Sp K a:xe]} \ras kein Bedeutungsunterschied

		\item \textbf{Komplementäre} Allophonie

		\ex. \textipa{[x]} vs. \textipa{[\c{c}]}
		
		\ex. \textipa{[bax]} vs. \textipa{[mI\c{c}]}
		
		\ex. *\textipa{[mIx]} vs. *\textipa{[ba\c{c}]}

		\item \textbf{Freie} Allophonie

		\ex. \textipa{[p\super{h}as]} vs. \textipa{[pas]}

		\item \textbf{Regionale und soziale} Variation (Unterart der freien Allophonie)

		\ex. \textipa{[PIS]} vs. \textipa{[PI\c{c}]}

	\end{itemize}
	
\end{itemize}

\end{frame}



%%%%%%%%%%%%%%%%%%%%%%%%%%%%%%%%%%%%%%%%%%%%%%%%%%%%%%%
%%%%%%%%%%%%%%%%%%%%%%%%%%%%%%%%%%%%%%%%%%%%%%%%%%%%%%%
%
\section{Phonetisch-phonologische Ebenen}
%
\frame{
\begin{multicols}{2}
\frametitle{~}
	\tableofcontents[currentsection]
\end{multicols}
}
%%%%%%%%%%%%%%%%%%%%%%%%%%%%%%%%%%%%%%%%%%%%%%%%%%%%%%%

\begin{frame}{Phonetisch-phonologische Ebenen}

	\begin{itemize}
		\item Unterscheidung von mindestens zwei Ebenen
		\item[$\rightarrow$] \textipa{[\;R a: t]} und \textipa{[\;R E: d 5]} (für $\langle$Rad$\rangle$ und $\langle$Räder$\rangle$)\\
		aber\\
		\textipa{[\;R a: t]} und \textipa{[\;R E: t @]} (für $\langle$Rat$\rangle$ und $\langle$Räte$\rangle$)
		\item[]
		\item[$\rightarrow$] Warum verstehen wir dasselbe, wenn wir\\
		\textipa{[h a: k @ n]} oder \textipa{[h a: k N]}\\
		hören?
		\item[]
		\item \textbf{Tiefenstruktur} (Deep Structure) vs. \textbf{Oberflächenstruktur} (Surface Structure)
	\end{itemize}
	
\end{frame}



%%%%%%%%%%%%%%%%%%%%%%%%%%%%%%%%%%%%%%%%%%%%%%%%%%%%%%%
%%%%%%%%%%%%%%%%%%%%%%%%%%%%%%%%%%%%%%%%%%%%%%%%%%%%%%%
%
\subsection{Tiefenstruktur (TS)}
%
%\frame{
%\begin{multicols}{2}
%\frametitle{~}
%	\tableofcontents[currentsection]
%\end{multicols}
%}
%%%%%%%%%%%%%%%%%%%%%%%%%%%%%%%%%%%%%%%%%%%%%%%%%%%%%%%

\begin{frame}{Tiefenstruktur (TS)}
	
\begin{itemize}
	\item \textbf{Zugrundeliegende abstrakte Repräsentation} $\rightarrow$ Phoneme \textipa{/ /}
	\item[]
	\item \textbf{Idiosynkratische} Form $\approx$ Nicht deriviert/abgeleitet
	\item[$\rightarrow$] Die TS-Form kann nicht durch Regeln abgeleitet werden, sie ist im Lexikon gespeichert.
	\item[]
	\item TS besteht aus Phonemen
	\item[$\rightarrow$] \textipa{/\;R a: t/}: TS-Form von $\langle$Rat$\rangle$\\
	\item[$\rightarrow$] \textipa{/\;R a: d/}: TS-Form von $\langle$Rad$\rangle$
	\item[$\rightarrow$] \textipa{/h a: k @ n/}: TS-Form von $\langle$Haken$\rangle$
\end{itemize}
		
\end{frame}



%%%%%%%%%%%%%%%%%%%%%%%%%%%%%%%%%%%%%%%%%%%%%%%%%%%%%%%

\begin{frame}
\frametitle{Tiefenstruktur (TS)}

\begin{itemize}
	\item \textipa{[t]} in \textipa{[\;R a: t]} (von \textipa{/\;R a: d/}) ist ableitbar
	\item \textipa{/d/} in \textipa{/\;R a: d/} ist idiosynkratisch
	\item[]
	\item \textipa{/t/} in \textipa{/\;R a: t/} ist idiosynkratisch
	\item[]
	\item Wenn das Deutsche ein neues Wort wie $\langle$Code$\rangle$ \textipa{[k @ U d]} entlehnen würde, würde dieses Wort früher oder später \gqq{eingedeutscht} werden.
	\item[$\rightarrow$] \textipa{[k O U t]} oder \textipa{[k o: t]} aber \gqq{des \textipa{[k O U d @ s]}} oder \gqq{des \textipa{[k o: t s]}} 
\end{itemize}

\end{frame}



%%%%%%%%%%%%%%%%%%%%%%%%%%%%%%%%%%%%%%%%%%%%%%%%%%%%%%%
%%%%%%%%%%%%%%%%%%%%%%%%%%%%%%%%%%%%%%%%%%%%%%%%%%%%%%%
%
\subsection{Oberflächenstruktur (OS)}
%
%\frame{
%\begin{multicols}{2}
%\frametitle{~}
%	\tableofcontents[currentsection]
%\end{multicols}
%}
%%%%%%%%%%%%%%%%%%%%%%%%%%%%%%%%%%%%%%%%%%%%%%%%%%%%%%%

\begin{frame}{Oberflächenstruktur (OS)}

\begin{itemize}
	\item Von der abstrakten phonembasierten TS wird die sog. Oberflächenstruktur mithilfe von vorhersagbaren (phonetisch-)phonologischen Regeln deriviert.
	\item[]
	\item OS entspricht der \textbf{tatsächlichen Realisierung} $\rightarrow$ Phone \textipa{[ ]}
	\item[]
	\item Demnach gibt es viele mögliche OS-Formen, darunter auch die sog.\\ 
\textbf{kanonische Aussprache} ($\approx$ Standardaussprache) $\rightarrow$ \textipa{[P e: b @ n]},\\
und die vielen möglichen\\
\textbf{umgangssprachlichen Formen} $\rightarrow$ \textipa{[P e: b n]}, \textipa{[P e: b m]}, \textipa{[P e: m]}
	\item[]
	\item Häufig wird zwischen phonologischen und phonetischen Prozessen unterschieden.
\end{itemize}

\end{frame}



%%%%%%%%%%%%%%%%%%%%%%%%%%%%%%%%%%%%%%%%%%%%%%%%%%%%%%%

\begin{frame}
\frametitle{{Oberflächenstruktur (OS)}}

\begin{itemize}
	\item Häufig wird zwischen phonologischen und phonetischen Prozessen unterschieden.
	\item[]
	\item \textbf{Phonetische Prozesse} $\rightarrow$ vom Sprachtempo und Stil abhängige Veränderungen
	\item[$\rightarrow$] Plosiveinsetzung: \textipa{/a m t/} $\rightarrow$ \textipa{[P a m p t]}
	\item[]
	\item \textbf{Phonologische Prozessen} $\rightarrow$ systematisch und obligatorisch auftretende Veränderungen
	\item[$\rightarrow$] \textit{Ich}-/\textit{Ach}-Laut-Wechsel \textipa{[b u: x]} (von \textipa{/b u: \c{c}/}) ist ableitbar
	\item[]
	\item Einen klaren Schnitt zwischen phonetischen und phonologischen Prozessen gibt es nicht!
	\item[$\rightarrow$] Sind g-Tilgung, Spirantisierung, Schwa-Tilgung, \dots phonetische oder phonologische Prozesse?
\end{itemize}

\end{frame}



%%%%%%%%%%%%%%%%%%%%%%%%%%%%%%%%%%%%%%%%%%%%%%%%%%%%%%%
%%%%%%%%%%%%%%%%%%%%%%%%%%%%%%%%%%%%%%%%%%%%%%%%%%%%%%%
%
\subsection{TS \& OS}
%
%%%%%%%%%%%%%%%%%%%%%%%%%%%%%%%%%%%%%%%%%%%%%%%%%%%%%%%
%\frame{
%\frametitle{~}
%	\tableofcontents[currentsection]
%}
%%%%%%%%%%%%%%%%%%%%%%%%%%%%%%%%%%%%%%%%%%%%%%%%%%%%%%%

\begin{frame}{TS \& OS}

\begin{itemize}
	\item TS \& OS sind \textbf{theoretische Abstraktionen} ($\approx$ keine Wahrheiten!), um die Regelhaftigkeiten auf der phonologischen Ebene erklären zu können.
	\item[]
	\item Kind erhält als \textbf{Input im Spracherwerb} OS-Formen wie: \textipa{[\;R a: t]} und \textipa{[\;R E: t @]}, \textipa{[\;R a: t]} und \textipa{[\;R E: d 5]}, \textipa{[b E t]} und \textipa{[b E t @ n]}, \textipa{[b a: t]} und \textipa{[b E: d 5]}, \textipa{[k I n t]} und \textipa{[k I n d 5]}
	\item[]
	\item Daraus erkennt das Kind,

	\begin{itemize}
		\item dass in einigen Wörtern \textipa{[d]} und \textipa{[t]} \textbf{systematisch} ausgetauscht werden (z.~B. $\langle$Rad$\rangle$, $\langle$Bad$\rangle$, $\langle$Kind$\rangle$),
		\item dass aber in anderen Wörtern \textipa{[t]} immer als \textipa{[t]} ausgesprochen wird  (z.~B. $\langle$Rat$\rangle$, $\langle$Bett$\rangle$).
	\end{itemize}
		
\end{itemize}

\end{frame}



%%%%%%%%%%%%%%%%%%%%%%%%%%%%%%%%%%%%%%%%%%%%%%%%%%%%%%%

\begin{frame}
	\frametitle{TS \& OS}
		
\begin{itemize}
	\item Daraus erkennt das Kind,
	
	\begin{itemize}
		\item dass in einigen Wörtern \textipa{[d]} und \textipa{[t]} \textbf{systematisch} ausgetauscht werden (z.~B. $\langle$Rad$\rangle$, $\langle$Bad$\rangle$, $\langle$Kind$\rangle$),
		\item dass aber in anderen Wörtern \textipa{[t]} immer als \textipa{[t]} ausgesprochen wird  (z.~B. $\langle$Rat$\rangle$, $\langle$Bett$\rangle$).
		\item[]
		\item Daraus leitet das Kind Folgendes ab:
		\item[$\rightarrow$] \textipa{/d/} $\rightarrow$ \textipa{[t]} am Ende des Wortes (bzw. der Silbe)!
		\item[]
		\item[] Aber nicht:
		\item[]
		\item[$\rightarrow$] \textipa{/t/} $\rightarrow$ \textipa{[d]}\\ (Andernfalls müsste der Plural von $\langle$Rat$\rangle$ \gqq{die \textipa{[\;R E: d @]}} heißen.)
	\end{itemize}
	
	\item[]
	\item Diese Regelhaftigkeit erweitert das Kind auf weitere Lauteinheiten bei weiterem Input $\rightarrow$ \textipa{/b d g z v Z/}	(sog. stimmhafte Obstruenten)
\end{itemize}

\end{frame}
	


%%%%%%%%%%%%%%%%%%%%%%%%%%%%%%%%%%%%%%%%%%%%%%%%%%%%%%%

\begin{frame}
\frametitle{TS \& OS}

\begin{table}
\centering 
		
\begin{tabular}{p{0.17\linewidth}p{0.15\linewidth}p{0.17\linewidth}p{0.15\linewidth}p{0.17\linewidth}}
	\hline
	\textbf{TS}\par \tiny{Phonologische\par Repräsentation\ (Lexikon)} & & \textbf{OS}\par \tiny{Phonetische\par Repräsentation\par (Standard)} & & \textbf{OS}\par \tiny{Phonetische\par Repräsentation\par (Umgangssprache)} \\
	\hline
	\textipa{/\;R a: d/} & $\begin{array}[c]{c}\rightarrow\end{array}$ & \textipa{[\;R a: t]} & & \\
	\hline
	\textipa{/\;R a: t/} & $\begin{array}[c]{c}\rightarrow\end{array}$ & \textipa{[\;R a: t]} & & \\
	\hline
	\textipa{/e: b @ n/} & $\begin{array}[c]{c}\rightarrow\end{array}$ & \textipa{[P e: b @ n]} & $\begin{array}[c]{c}\rightarrow\end{array}$ & \textipa{[P e: b m]}\\
	\hline
	& \small{Phonologische\par Prozesse} &  & \small{Phonetische\par Prozesse} & \\
	\hline		
\end{tabular}

\caption{TS $\rightarrow$ OS} 
\end{table}

\begin{itemize}
	\item Die Abstraktion (s. Tabelle) impliziert eine gewisse zeitliche Abfolge, die es in der Realität nicht gibt. Es handelt sich um eine theoretische Abstraktion, die notwendig ist, um Phänomene zu erfassen!	
\end{itemize}
			
\end{frame}


%%%%%%%%%%%%%%%%%%%%%%%%%%%%%%%%%%%%%%%%%%%%%%%%%%%%%%%
%%%%%%%%%%%%%%%%%%%%%%%%%%%%%%%%%%%%%%%%%%%%%%%%%%%%%%%
%
\section{Phonetisch/phonologische Prozesse}
%
%%%%%%%%%%%%%%%%%%%%%%%%%%%%%%%%%%%%%%%%%%%%%%%%%%%%%%%
%\frame{
%\frametitle{~}
%	\tableofcontents[currentsection]
%}
%%%%%%%%%%%%%%%%%%%%%%%%%%%%%%%%%%%%%%%%%%%%%%%%%%%%%%%

\begin{frame}{Phonetisch/phonologische Prozesse}

\begin{itemize}
	\item Tilgung von Segmenten
	\item[]
	\item Hinzufügung von Segmenten
	\item[]
	\item Veränderung von Segmenten
	\item[]
	\item Allgemeine Notation: A \ras B / C \underline{\quad} D
	\item[] Ein Segment A im Input wird zu einem Segment B im Output in einem Kontext (\gqq{/}), in dem C \textit{vor} und D \textit{nach} A vorkommt. 
\end{itemize}

\end{frame}



%%%%%%%%%%%%%%%%%%%%%%%%%%%%%%%%%%%%%%%%%%%%%%%%%%%%%%%
%%%%%%%%%%%%%%%%%%%%%%%%%%%%%%%%%%%%%%%%%%%%%%%%%%%%%%%
%
\subsection{Tilgung von Segmenten}
%
%%%%%%%%%%%%%%%%%%%%%%%%%%%%%%%%%%%%%%%%%%%%%%%%%%%%%%%
%\frame{
%\frametitle{~}
%	\tableofcontents[currentsection]
%}
%%%%%%%%%%%%%%%%%%%%%%%%%%%%%%%%%%%%%%%%%%%%%%%%%%%%%%%

\begin{frame}{Tilgung von Segmenten}

\begin{itemize}
	\item \textbf{\textipa{/@/}-Tilgung}:
	
	\begin{itemize}
		\item Fakultativ
		\item Regel: \textipa{/@/} \ras $\emptyset$ / X \underline{\quad} $\{$[sonorant]; absoluter Auslaut$\}$

		\ex. $\langle$gehen$\rangle$: \textipa{/ge:.@n/} \ras \textipa{[ge:n]}
		
		\ex. $\langle$kaufe$\rangle$: \textipa{/k\t{aU}.f@/} \ras \textipa{[k\t{aU}f]}
		
		\ex. $\langle$Kumpel$\rangle$: \textipa{/kUm.p@l/} \ras \textipa{[kUm.p\textsyllabic{l}]}

	\end{itemize}

	\item \textbf{\textipa{/g/}-Tilgung}:
	
	\begin{itemize}
		\item Obligatorisch
		\item Regel: \textipa{/g/} \ras $\emptyset$ / [nasal, velar] \underline{\quad} $]_\sigma$
		\ex. $\langle$Tilgung$\rangle$: \textipa{[tIl.gUNg]} \ras \textipa{[tIl.gUN]}
		
	\end{itemize}
			
\end{itemize}

\end{frame}


%%%%%%%%%%%%%%%%%%%%%%%%%%%%%%%%%%%%%%%%%%%%%%%%%%%%%%%

\begin{frame}
\frametitle{Tilgung von Segmenten}

\begin{itemize}
	\item \textbf{Geminatenreduktion}:

	\begin{itemize}
		\item Fakultativ
		\item Regel: XX \ras X / A \underline{\quad} B

		\ex. $\langle$Enttäuschung$\rangle$: \textipa{/Ent.t\t{OI}.SUng/} \ras \textipa{[PEn\.t\t{OI}.SUN]}
		
		\ex. $\langle$Schifffahrt$\rangle$: \textipa{/SIf.fa:\;Rt/} \ras \textipa{[SI\.fa:\;Rt]}
		
		\ex. ABER $\langle$Zoooper$\rangle$: \textipa{/\t{ts}o:.o.p@\;R/} \ras \textipa{[\t{ts}o:.Po.p5]}
		
	\end{itemize}
	
\end{itemize}

\end{frame}



%%%%%%%%%%%%%%%%%%%%%%%%%%%%%%%%%%%%%%%%%%%%%%%%%%%%%%%
%%%%%%%%%%%%%%%%%%%%%%%%%%%%%%%%%%%%%%%%%%%%%%%%%%%%%%%
%
\subsection{Hinzufügung von Segmenten}
%
%\frame{
%\begin{multicols}{2}
%\frametitle{~}
%	\tableofcontents[currentsection]
%\end{multicols}
%}
%%%%%%%%%%%%%%%%%%%%%%%%%%%%%%%%%%%%%%%%%%%%%%%%%%%%%%%

\begin{frame}{Hinzufügung von Segmenten}

\begin{itemize}
	\item Allgemeine Regel: $\emptyset$ \ras X / A \underline{\quad} B
	\item[]
	\item \textbf{Plosiveinsetzung}:
	
	\begin{itemize}
		\item Fakultativ

		\ex. $\langle$Amt$\rangle$: \textipa{/amt/} \ras \textipa{[Pampt]}
		
		\ex. $\langle$Gans$\rangle$: \textipa{/gans/} \ras \textipa{[gants]}

	\end{itemize}

	\item \textbf{Knacklauteinsetzung}:

	\begin{itemize}
		\item (Fast) Obligatorisch
		\item Plosiveinsetzung
		\item Regel: $\emptyset$ \ras \textipa{[P]} / 
		$\{$\#; \textprimstress$_\sigma[\}$ 
\underline{\quad} V

		\ex. $\langle$Beamte$\rangle$: \textipa{/b@.\textprimstress am.t@/} \ras \textipa{[b@.\textprimstress Pam.t@]}
		
		\ex. $\langle$Apfel$\rangle$: \textipa{/a\t{pf}@l/} \ras \textipa{[Pa\t{pf}@l]}
		
		\ex. ABER $\langle$gehen$\rangle$: \textipa{/\textprimstress ge:.@n/} $\nrightarrow$ \textipa{[\textprimstress ge:.P@n]} sondern: \textipa{[\textprimstress ge:.@n]}

	\end{itemize}
			
\end{itemize}

\end{frame}



%%%%%%%%%%%%%%%%%%%%%%%%%%%%%%%%%%%%%%%%%%%%%%%%%%%%%%%
%%%%%%%%%%%%%%%%%%%%%%%%%%%%%%%%%%%%%%%%%%%%%%%%%%%%%%%
%
\subsection{Veränderung von Segmenten (durch Assimilation)}
%
%\frame{
%\begin{multicols}{2}
%\frametitle{~}
%	\tableofcontents[currentsection]
%\end{multicols}
%}
%%%%%%%%%%%%%%%%%%%%%%%%%%%%%%%%%%%%%%%%%%%%%%%%%%%%%%%

\begin{frame}{Veränderung von Segmenten (durch Assimilation)}

\begin{itemize}
	\item \textbf{Regressive velare Nasalassimilation}

	\begin{itemize}
		\item Obligatorisch (innerhalb des phonologischen Wortes)
		\item Regel: \textipa{/n/} \ras \textipa{[N]} /  \underline{\quad} [velar, plosiv]

		\ex. $\langle$Führung$\rangle$: \textipa{/fy:.\;RUng/} \ras \textipa{[fy:.\;RUNg]} (nach g-Tilgung \ras \textipa{[fy:.\;RUN]})
		
		\ex. $\langle$Bank$\rangle$: \textipa{/bank/} \ras \textipa{[baNk]}
		
		\ex. ABER $\langle$ungern$\rangle$: \textipa{/Un.gE\;Rn/} \ras \textipa{[PUn.gE\;Rn]} oder fakulativ \textipa{[PUN.gE\;Rn]}

	\end{itemize}
	
	\item[]
	\item \textbf{(Allgemeine) regressive Nasalassimilation}:

	\begin{itemize}
		\item Fakultativ
		\item Regel: [nasal, Art.Ort: Y] \ras [nasal, Art.Ort: X] /  \underline{\quad} [obstruent, Art.Ort: X]

		\ex. $\langle$fünf$\rangle$: \textipa{/fYnf/} \ras \textipa{[fYmf]}

	\end{itemize}		

\end{itemize}

\end{frame}



%%%%%%%%%%%%%%%%%%%%%%%%%%%%%%%%%%%%%%%%%%%%%%%%%%%%%%%

\begin{frame}
{Veränderung von Segmenten (durch Assimilation)}

\begin{itemize}
	\item \textbf{Progressive Nasalassimilation}:

	\begin{itemize}
		\item Fakultativ
		\item Regel: [nasal, Art.Ort: Y] $\rightarrow$ [nasal, Art.Ort: X] /  [obstruent, Art.Ort: X] \underline{\quad} 

		\ex. $\langle$Haken$\rangle$: \textipa{/ha:k@n/} \ras 
		\textipa{[ha:k\textsyllabic{n}]} \ras \textipa{[ha:k\textsyllabic{N}]}
		
		\ex. $\langle$Schuppen$\rangle$: \textipa{/SUp@n/} \ras 
		\textipa{[SU\.p\textsyllabic{n}]} \ras \textipa{[SU\.p\textsyllabic{m}]}

	\end{itemize}

	\item[]
	\item \textbf{\textipa{[\c{c}]/[x]}-Alternation (Dorsale Assimilation)}

	\begin{itemize}
		\item Obligatorisch
		\item Regel: \textipa{/\c{c}/} $\rightarrow$ \textipa{[x]} / Hinterer Vokal \underline{\quad}

		\ex. $\langle$mich$\rangle$: \textipa{/mI\c{c}/} $\rightarrow$ \textipa{[mI\c{c}]}
		
		\ex. $\langle$Buch$\rangle$: \textipa{/bU\c{c}/} $\rightarrow$ \textipa{[bUx]}
		
		\ex. $\langle$Elch$\rangle$: \textipa{/El\c{c}/} $\rightarrow$ \textipa{[PEl\c{c}]}

	\end{itemize}		

\end{itemize}

\end{frame}



%%%%%%%%%%%%%%%%%%%%%%%%%%%%%%%%%%%%%%%%%%%%%%%%%%%%%%%

\begin{frame}
\frametitle{Veränderung von Segmenten (durch Assimilation)}

\begin{itemize}
	\item \textbf{\textipa{/g/}-Spirantisierung}

	\begin{itemize}
		\item Fakultativ (dialektal)
		\item Regel: \textipa{/g/} \ras \textipa{/\c{c}/} / V\underline{\quad} $]_\sigma$

	\ex. $\langle$sagst$\rangle$: \textipa{/za:gst/} $\rightarrow$ \textipa{[za:xst]}
	
	\ex. $\langle$freudig$\rangle$: \textipa{/f\;R\t{OI}.dIg/} $\rightarrow$ \textipa{[f\;R\t{OI}.dI\c{c}]}

	\end{itemize}

	\item[]
	\item \textbf{\textipa{/\;R/}-Vokalisierung}
	
	\begin{itemize}
		\item Fakultativ -- Obligatorisch
		\item Regel: \textipa{/\;R/} \ras \textipa{[5]} / V\underline{\quad} $]_\sigma$

		\ex. $\langle$Ohr$\rangle$: \textipa{/o:\;R/} $\rightarrow$ \textipa{[Po:5]}

		\ex. $\langle$fern$\rangle$: \textipa{/fE\;Rn/} $\rightarrow$ \textipa{[fE5n]}

		\ex. $\langle$Lehrer$\rangle$: \textipa{/le:.\;R@\;R/} $\rightarrow$ \textipa{[le:.\;R@5]} (nach Schwa-Tilgung $\rightarrow$ \textipa{[le:.\;R5]}

	\end{itemize}

\end{itemize}

\end{frame}



%%%%%%%%%%%%%%%%%%%%%%%%%%%%%%%%%%%%%%%%%%%%%%%%%%%%%%%

\begin{frame}
{Veränderung von Segmenten (durch Assimilation)}

\begin{itemize}
	\item \textbf{Auslautverhärtung}

	\begin{itemize}
		\item Obligatorisch
		\item Regel: /obstruent, stimmhaft/ \ras [obstruent, stimmlos] / \underline{\quad} $]_\sigma$

		
		\ex. $\langle$Bad$\rangle$: \textipa{/ba:d/} \ras \textipa{[ba:t]}
		
		\ex. ABER $\langle$Bäder$\rangle$: \textipa{/bE:.d@\;R/} \ras \textipa{[bE:.d5]}
		
		\ex. $\langle$oliv$\rangle$: \textipa{/oli:v/} \ras \textipa{[Po.li:f]}
		
		\ex. ABER $\langle$Olive$\rangle$: \textipa{/oli:v@/} \ras \textipa{[Po.li:.v@]}
		
		\ex. $\langle$Endspurt$\rangle$: \textipa{/End.SpU\;Rt/} \ras \textipa{[PEnt.SpU\;Rt]}
		
		\ex. ABER $\langle$Ende$\rangle$: \textipa{/En.d@/} \ras \textipa{[PEn.d@]}

	\end{itemize}

\end{itemize}

\end{frame}



%%%%%%%%%%%%%%%%%%%%%%%%%%%%%%%%%%%%%%%%%%%%%%%%%%%%%%%
%%%%%%%%%%%%%%%%%%%%%%%%%%%%%%%%%%%%%%%%%%%%%%%%%%%%%%%
%
\subsection{Reihenfolge der Prozesse}
%
%\frame{
%\begin{multicols}{2}
%\frametitle{~}
%	\tableofcontents[currentsection]
%\end{multicols}
%}
%%%%%%%%%%%%%%%%%%%%%%%%%%%%%%%%%%%%%%%%%%%%%%%%%%%%%%%

\begin{frame}{Reihenfolge der Prozesse}

\begin{itemize}
	\item Reihenfolge der Prozesse spielt eine wichtige Rolle!

	\begin{block}{Feeding}
	Wenn ein Prozess die kontextuellen Bedingungen für einen weiteren Prozess \textbf{schafft}.	

	\end{block}

	\ex. $\langle$Haken$\rangle$: \textipa{/ha:k@n/} \ras \textipa{[ha:k\textsyllabic{n}]} \ras \textipa{[ha:k\textsyllabic{N}]}

	\begin{block}{Bleeding}
	Wenn ein Prozess die kontextuellen Bedingungen für einen weiteren Prozess \textbf{zerstört}.
	\end{block}

	\ex. $\langle$Gesang$\rangle$: \textipa{/g@.zang/} \ras \textipa{[g@.zaNg]} \ras \textipa{[g@.zaN]} $\nrightarrow$ \textipa{[g@.zaNk]}
	
\end{itemize}

\end{frame}