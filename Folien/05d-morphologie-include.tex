%%%%%%%%%%%%%%%%%%%%%%%%%%%%%%%%%%%%%%%%%%%%%%%%
%% Compile the master file!
%% 		Slides: Antonio Machicao y Priemer
%% 		Course: GK Linguistik
%%%%%%%%%%%%%%%%%%%%%%%%%%%%%%%%%%%%%%%%%%%%%%%%

%%%%%%%%%%%%%%%%%%%%%%%%%%%%%%%%%%%%%%%%%%%%%%%%%%%%
%%%             Metadata                         %%%
%%%%%%%%%%%%%%%%%%%%%%%%%%%%%%%%%%%%%%%%%%%%%%%%%%%%      

\title{Grundkurs Linguistik}

\subtitle{Morphologie IV: Flexion und Typologie}

\author[aMyP]{
	{\small Antonio Machicao y Priemer}
%	\\
%	{\footnotesize \url{http://www.linguistik.hu-berlin.de/staff/amyp}\\
%	\href{mailto:mapriema@hu-berlin.de}{mapriema@hu-berlin.de}}
}

\institute{Institut für deutsche Sprache und Linguistik}

%%%%%%%%%%%%%%%%%%%%%%%%%      
\date{ }
%\publishers{\textbf{6. linguistischer Methodenworkshop \\ Humboldt-Universität zu Berlin}}

%\hyphenation{nobreak}


%%%%%%%%%%%%%%%%%%%%%%%%%%%%%%%%%%%%%%%%%%%%%%%%%%%%
%%%             Preamble's End                   %%%
%%%%%%%%%%%%%%%%%%%%%%%%%%%%%%%%%%%%%%%%%%%%%%%%%%%%      


%%%%%%%%%%%%%%%%%%%%%%%%%      
\huberlintitlepage[22pt]
\iftoggle{toc}{
\frame{
\begin{multicols}{2}
	\frametitle{Inhaltsverzeichnis}\tableofcontents
	%[pausesections]
\end{multicols}
	}
	}


%%%%%%%%%%%%%%%%%%%%%%%%%%%%%%%%%%%
%%%%%%%%%%%%%%%%%%%%%%%%%%%%%%%%%%%
\section{Morphologie III}
%%%%%%%%%%%%%%%%%%%%%%%%%%%%%%%%%%%

\begin{frame}
\frametitle{Begleitlektüre}

\begin{itemize}
	
	\item Flexion:
	\begin{itemize}
		\item \textbf{obligatorisch:}
		\item[] \citet[51--53]{Abramowski2016a}
		\item[] \citet[Kap. 8]{Luedeling2009a}		
		
		\item \textbf{optional:}
		\item[] \citet[Kap. 2, S. 21--29]{Meibauer&Co07a}
	\end{itemize}

	\item Typologie:
	\begin{itemize}
		\item tbd
	\end{itemize}

\end{itemize}

\end{frame}


%%%%%%%%%%%%%%%%%%%%%%%%%%%%%%%%%%
%%%%%%%%%%%%%%%%%%%%%%%%%%%%%%%%%%
\subsection{Flexion}

\iftoggle{sectoc}{
	\frame{
		%		\begin{multicols}{2}
		\frametitle{~}
		\tableofcontents[currentsubsection,subsubsectionstyle=hide]
		%		\end{multicols}
	}
}

%%%%%%%%%%%%%%%%%%%%%%%%%%%%%%%%%%%

\begin{frame}
\frametitle{Flexion}

\begin{itemize}
	\item Flexion \ras Bildung von Wortformen aus Stämmen
	\item Sprachspezifisch und wortartspezifisch werden verschiedene morphosyntaktische Flexionskategorien markiert (per Flexionsmorphem oder z.\,B. per Stammabwandlung)
	\item[]
	\item \textbf{Synthetische} Wortform: Flexionsstamm $+$ Flexionsaffix
	\item \textbf{Analytische} Wortformen: mehrere Elemente werden verwendet um eine Wortform abzubilden.
	
	\ea les $+$ (en) \ras las (Stammabwandlung per Ablaut)
	\z
	
	\ea kauf $+$ (en) \ras kauf $+$ tet (Affigierung \ras  synthetische Form)
	\z
	
	\ea verwend $+$ (en) \ras wird verwendet haben (analytische Form)
	\z
	
\end{itemize}


\end{frame}



%%%%%%%%%%%%%%%%%%%%%%%%%%%%%%%%%%%

\begin{frame}
\frametitle{Flexion}

\begin{itemize}
\item Flexionsstämme können selbst morphologisch komplex sein (ver$+$wend(en)).
\item Ein morphologisch nicht komplexer Stamm heißt \gqq{Wurzel} (kauf(en)).
\item[] 
\item Bei der Flexion wird unterschieden zwischen:

\begin{itemize}
	\item der \textbf{Deklination} von Nomina und anderer nominaler Kategorien (wozu auch Adjektive, Pronomina und Artikel gehören)
	\item[]
	\item der \textbf{Konjugation} von Verben 
	\item[]
	\item Ob die \textbf{Komparation} von Adjektiven -- mit den Kategorien Positiv, Komparativ und Superlativ -- zur Flexion oder zur Wortbildung gehört, ist umstritten.
\end{itemize}

\end{itemize}

\end{frame}



%%%%%%%%%%%%%%%%%%%%%%%%%%%%%%%%%%%

\begin{frame}
\frametitle{Flexion}

\begin{itemize}
\item Die nicht-flektierbaren Wortarten sind:

\begin{itemize}
\item[]
\item Adverbien
\item[]
\item Partikeln
\item[]
\item Präpositionen
\item[]
\item Konjunktionen
\end{itemize}
\end{itemize}


\end{frame}



%%%%%%%%%%%%%%%%%%%%%%%%%%%%%%%%%%%
%%%%%%%%%%%%%%%%%%%%%%%%%%%%%%%%%%%

\subsubsection{Deklination}
%\frame{
%\frametitle{~}
%	\tableofcontents[currentsection]
%}


%%%%%%%%%%%%%%%%%%%%%%%%%%%%%%%%%%%

\begin{frame}
\frametitle{Deklination}

\begin{itemize}
\item Deklination umfasst die Bildung von Wortformen bei nominalen Kategorien.
\item[]
\item \textbf{Substantive} deklinieren nach:

\begin{itemize}
\item[]
\item \textbf{Numerus:} Singular, Plural
\item[]
\item \textbf{Kasus:} Nominativ, Genitiv, Dativ, Akkusativ
\item[]
\item Substantive haben ein \textbf{inhärentes Genus}, d.\,h. sie flektieren nicht nach Genus. Die \textbf{Stärke} bei Nomina ist auch inhärent.
\end{itemize}

\end{itemize}

\end{frame}



%%%%%%%%%%%%%%%%%%%%%%%%%%%%%%%%%%%

\begin{frame}
\frametitle{Deklination}

\begin{itemize}
\item Beispiel:

\begin{itemize}
\item \textbf{Stark:} Maskulina und Neutra mit Nullendung im Nominativ und s-Genitiv

\ea Tisch, Fenster
\z

\item \textbf{Schwach:} Maskulina au\ss{}er im Nominativ stets mit -(e)n

\ea Held, Nachbar
\z

\item \textbf{Gemischt:} Maskulina und Neutra stark im Singular, schwach im Plural

\ea Staat, Ende
\z

\item \textbf{Unveränderliche} Feminina: endungslos im Singular und mit konsequenter Markierung im Plural

\ea Frau, Hand, Katze, Nadel
\z

\end{itemize}
\end{itemize}


\end{frame}



%%%%%%%%%%%%%%%%%%%%%%%%%%%%%%%%%%%

\begin{frame}
\frametitle{Deklination}

\begin{itemize}
\item \textbf{Paradigma:}\\
Die Gesamtheit der Flexionsformen eines Wortes (egal welcher Wortart) bilden sein Flexionsparadigma.
\end{itemize}

\begin{table}
\centering

\begin{tabular}{p{1.8cm}|p{1.8cm}|p{1.8cm}|p{1.8cm}|p{1.8cm}}
& \textbf{Nom} & \textbf{Akk} & \textbf{Dat} & \textbf{Gen}\\
\hline
\textbf{Singular} & Tisch & Tisch & Tisch(e) & Tisches\\
\hline
\textbf{Plural} & Tische & Tische & Tischen & Tische\\

\end{tabular}

\end{table}

\end{frame}



%%%%%%%%%%%%%%%%%%%%%%%%%%%%%%%%%%%

\begin{frame}
\frametitle{Deklination}

\begin{itemize}
\item \textbf{Adjektive} deklinieren nach:

\begin{itemize}
\item[]
\item \textbf{Numerus:} Singular, Plural
\item []
\item \textbf{Kasus:} Nominativ, Genitiv, Dativ, Akkusativ
\item[]
\item \textbf{Genus:} Maskulinum, Femininum, Neutrum
\item[]
\item \textbf{Stärke:} stark, schwach, gemischt (ob starke oder schwache Flexionsendungen beim Adjektiv verwendet werden, hängt vom Artikel ab)
\item[]
\item \textbf{Grad:} positiv (schön), komparativ (schöner), superlativ (am schönsten); ob Adjektive nach Grad flektieren, ist umstritten.
\end{itemize}

\end{itemize}


\end{frame}



%%%%%%%%%%%%%%%%%%%%%%%%%%%%%%%%%%%

\begin{frame}
\frametitle{Deklination}

\begin{itemize}
\item Beispiel Stärke

\begin{itemize}
\item \textbf{Stark:} ohne Artikel

\ea schön\underline{es} Wetter, schön\underline{er} Tag, schön\underline{e} Frau
\z

\item \textbf{Schwach:} nach bestimmten Artikeln oder einer entsprechend deklinierten Einheit

\ea das gut\underline{e} Kind, dieser schön\underline{e} Tag, jede schön\underline{e} Frau
\z

\item \textbf{Gemischt:} nach unbestimmten Artikeln oder einer entsprechend deklinierten Einheit

\ea ein gut\underline{es} Kind, ein schön\underline{er} Tag, keine schön\underline{e} Frau
\z

\end{itemize}

\end{itemize}

\end{frame}



%%%%%%%%%%%%%%%%%%%%%%%%%%%%%%%%%%%

\begin{frame}
\frametitle{Deklination}

\begin{itemize}
\item Im Deutschen wirken zur Flexionsanzeige Artikel, Adjektiv und Substantiv zusammen (= Wortgruppenflexion), da Artikel und Adjektiv mit dem Nomen in Numerus, Kasus und Genus \textbf{kongruieren} müssen, d.h. sie müssen die gleichen Numerus-, Kasus-, und Genusmerkmale aufweisen.

\eal 
\ex ein schöner Hund -- schöne Hunde -- des schönen Hundes
\ex ein schlaues Buch -- schlaue Bücher -- des schlauen Buches
\zl

\item U.\,U. wird nur an einem Element Kasus und Numerus der gesamten Phrase ersichtlich:

\eal 
\ex Der dicke Balken muss ersetzt werden.
\ex Der Architekt ordnete den Ersatz der dicken Balken an.
\zl

\end{itemize}


\end{frame}


%%%%%%%%%%%%%%%%%%%%%%%%%%%%%%%%%%%
%%%%%%%%%%%%%%%%%%%%%%%%%%%%%%%%%%%
\subsubsection{Konjugation}
%\frame{
%\frametitle{~}
%	\tableofcontents[currentsection]
%}


%%%%%%%%%%%%%%%%%%%%%%%%%%%%%%%%%%%

\begin{frame}
\frametitle{Konjugation}

\begin{itemize}
\item Bei der \textbf{Verbflexion} spricht man von \textbf{Konjugation}.
\item[]
\item Zunächst unterscheidet man zwischen \textbf{finiten und infiniten} Verbformen.
\item[]
\item \textbf{Infinite} Verbformen sind unveränderlich, d.h. egal in welchem Kontext sie stehen, sehen sie immer gleich aus.
\item[]
\item Dazu gehören: Infinitiv, Partizip I, Partizip II

\ea essen, essend, gegessen
\z

\end{itemize}


\end{frame}



%%%%%%%%%%%%%%%%%%%%%%%%%%%%%%%%%%%

\begin{frame}
\frametitle{Konjugation}

\begin{itemize}
\item Finite Verbformen sind veränderlich. Sie verändern ihre Form nach: 

\begin{itemize}
\item[]
\item \textbf{Person:} 1.,2.,3.
\item[]
\item \textbf{Numerus:} Singular, Plural
\item[]
\item \textbf{Modus:} Indikativ, Konjunktiv, Imperativ
\item[]
\item \textbf{Tempus:} Präsens, Präteritum, Perfekt, Plusquamperfekt, Futur I/ II
\item[]
\item \textbf{Genus verbi:} Aktiv, Passiv
\end{itemize}

\end{itemize}


\end{frame}



%%%%%%%%%%%%%%%%%%%%%%%%%%%%%%%%%%%

\begin{frame}
\frametitle{Konjugation}

\begin{itemize}
\item Beispiel Stärke:

\begin{itemize}
\item[]
\item \textbf{Starke} Konjugation: Vokalwechsel, Ablaut

\ea essen, aß, gegessen/ rufen, rief, gerufen
\z

\item \textbf{Schwache} Konjugation: immer mit -te im Präteritum, im mit -t im Partizip Perfekt

\ea kaufen, kaufte, gekauft/ arbeiten, arbeitete, gearbeitet
\z

\item \textbf{Gemischte} Konjugation: Vokalwechsel, immer mit -te im Präteritum, immer mit -t im Partizip Perfekt

\ea wissen, wusste, gewusst/ kennen, kannte, gekannt
\z

\end{itemize}

\end{itemize}


\end{frame}



%%%%%%%%%%%%%%%%%%%%%%%%%%%%%%%%%%

\begin{frame}
\frametitle{Konjugation}

\begin{itemize}
\item Es gibt verschiedene Mittel, die Flexion bei Verben anzuzeigen. 

\begin{itemize}
\item[]
\item \textbf{Flexionsaffixe} (\textit{nehm -- nehmt}) 
\item[]
\item \textbf{Ablautbildung} (fahr -- fuhr) mit anschließender \textbf{Umlautbildung} (führest) 
\item[]
\item Änderungen am \textbf{Konsonanten} im Stamm (\textit{bringen -- gebracht})
\item[]
\item \textbf{analytische} Mittel (Kombination mehrerer Wörter: ist abgeholt) 
\item[]
\item Oft werden diese Mittel miteinander kombiniert\\
(s. gebracht: Zirkumfix \ab{ge-{\dots}-t} $+$ Ablautbildung \ab{ie \ras a} $+$ Konsonantenänderung \\ab{ng \ras ch})
\end{itemize}

\end{itemize}


\end{frame}



%%%%%%%%%%%%%%%%%%%%%%%%%%%%%%%%%%%

\begin{frame}
\frametitle{Konjugation}

\begin{itemize}
\item Von \textbf{Suppletion} spricht man, wenn bei bestimmten grammatischen Merkmalen ein völlig anderer Stamm benutzt wird:

\vspace{1em}

\ea sein -- bist -- war
\z

\ea gut -- besser -- am besten
\z

%\vspace{6em}	

%	\item \textbf{ÜB 5, 6 \& 7}

\end{itemize}


\end{frame}



%%%%%%%%%%%%%%%%%%%%%%%%%%%%%%%%%%%
%%%%%%%%%%%%%%%%%%%%%%%%%%%%%%%%%%
\subsection{Typologie}

\iftoggle{sectoc}{
	\frame{
		%		\begin{multicols}{2}
		\frametitle{~}
		\tableofcontents[currentsubsection,subsubsectionstyle=hide]
		%		\end{multicols}
	}
}

%%%%%%%%%%%%%%%%%%%%%%%%%%%%%%%%%%
\begin{frame}
\frametitle{Einführung}

\begin{itemize}
	\item Unterscheidung von Sprachtypen nach der Art der Realisierung der Flexion (welche Flexionskategorien und wie werden angezeigt):
	
	\begin{itemize}
		\item analytische:
		
		\begin{itemize}
			\item \textbf{isolierend}: jedes morphologische Merkmal wird durch ein separates freies Morphem realisiert
		\end{itemize}
		
		\item[]
		\item synthetische:
		
		\begin{itemize}
			\item \textbf{agglutinierend}
			\item \textbf{fusionierend} (flektierend)
			\item \textbf{polysynthetisch} (inkorporierend)
		\end{itemize}
		
		\item[]
		\item Mischformen (die meisten Sprachen)
	\end{itemize}
\end{itemize}


\end{frame}


%%%%%%%%%%%%%%%%%%%%%%%%%%%%%%%%%%
%%%%%%%%%%%%%%%%%%%%%%%%%%%%%%%%%%
\subsubsection{Isolierende Sprachen}
%\frame{
%\frametitle{~}
%	\tableofcontents[currentsection]
%}


%%%%%%%%%%%%%%%%%%%%%%%%%%%%%%%%%%
\begin{frame}
\frametitle{Isolierende Sprachen}

\begin{itemize}
	\item Grammatische Beziehungen zwischen Wörtern im Satz durch selbständige, \textbf{syntaktische Formenelemente} realisiert
	\item[] \ras keine gebundenen Morpheme
	\item Vietnamesisch, Chinesisch, westafrikanische Sprachen
	
	\item[] Vietnamesisch:
	
%\ea
%\gll \foreignlanguage{vietnamese}{khi} \foreignlanguage{vietnamese}{tôi} \foreignlanguage{vietnamese}{đến} \foreignlanguage{vietnamese}{nhà} \foreignlanguage{vietnamese}{bạn} \foreignlanguage{vietnamese}{tôi} \foreignlanguage{vietnamese}{chúng} \foreignlanguage{vietnamese}{tôi} \foreignlanguage{vietnamese}{bắt đấu} \foreignlanguage{vietnamese}{làm} \foreignlanguage{vietnamese}{bài} \\
%als 1P komm Haus Freund 1P PL 1P anfangen tun Übung \\
%\mytrans{Als ich zum Haus meines Freundes kam, begannen wir, Übungen zu machen.}
%\z

\end{itemize}


\end{frame}

%%%%%%%%%%%%%%%%%%%%%%%%%%%%%%%%%%
\begin{frame}
\frametitle{Isolierende Sprachen}

\begin{itemize}
	\item Auch im Deutschen oder Englischen gibt es Formen der Isolation, etwa Auxiliare
	
	\begin{itemize}
		\item[]
		\item im Deutschen aber mit Flexion verbunden, im Englischen oft ohne Flexion
		
		\eal 
			\ex Ich werd-e gehen.
			\ex Wir werd-en gehen.
			\ex I/you/(s)he/we/they will go.
		\zl
		
	\end{itemize}
\end{itemize}


\end{frame}


%%%%%%%%%%%%%%%%%%%%%%%%%%%%%%%%%%
%%%%%%%%%%%%%%%%%%%%%%%%%%%%%%%%%%
\subsubsection{Agglutinierende Sprachen}
%\frame{
%\frametitle{~}
%	\tableofcontents[currentsection]
%}


%%%%%%%%%%%%%%%%%%%%%%%%%%%%%%%%%%
\begin{frame}
\frametitle{Agglutinierende Sprachen}

\begin{itemize}
	\item Grammatische und lexikalische Morpheme mit jeweils einfachen Bedeutungen werden \textbf{aneinandergereiht}
	\item[]
	\item \textbf{1:1-Zuordnung} von Morphem und Bedeutung/Funktion
	\item[]
	\item Resultat: hochkomplexe Wörter mit zahlreichen Morphemen
	\item[]
	\item Türkisch, Finnisch, Ungarisch, Bantu-Sprachen
	
	\ea
	\gll	çalış - tIr - Il - mA - mAlI - ymIş\\
			arbeit - Verursachung - Passiv - Negation - Obligation - Evidenz \\
			çalıştırılmamalıymış \\
			\mytrans{anscheinend sollte man ihn nicht zur Arbeit veranlassen.} \\
	\z
			
%% @ee Leipzig glossing rules

\end{itemize}


\end{frame}


%%%%%%%%%%%%%%%%%%%%%%%%%%%%%%%%%%
%%%%%%%%%%%%%%%%%%%%%%%%%%%%%%%%%%
\subsubsection{Fusionierende Sprachen}
%\frame{
%\frametitle{~}
%	\tableofcontents[currentsection]
%}


%%%%%%%%%%%%%%%%%%%%%%%%%%%%%%%%%%
\begin{frame}
\frametitle{Fusionierende Sprachen}

\begin{itemize}
	\item auch \textbf{flektierende} Sprachen genannt
	\item[]
	\item Die Morpheme sind oft \textbf{polysem} (ein Flexionsmorphem trägt verschiedene grammatische Informationen).
	\item[]
	\item Darüber hinaus kann ein Flexionsmorphem gleichlautend mit einem funktional anderen sein (\zB\ -en), \dash es kommt zu \textbf{Allomorphie}.
	\item[]
	\item Bestimmte morphologische Prozesse werden mehrfach markiert (\zB bei der Pluralbildung: Affigierung plus Stammvokaländerung).
	\item[]
	\item Zu den flektierenden Sprachen gehören die indogermanischen Sprachen.
\end{itemize}


\end{frame}


%%%%%%%%%%%%%%%%%%%%%%%%%%%%%%%%%%
%%%%%%%%%%%%%%%%%%%%%%%%%%%%%%%%%%
\subsubsection{Polysynthetische Sprachen}
%\frame{
%\frametitle{~}
%	\tableofcontents[currentsection]
%}


%%%%%%%%%%%%%%%%%%%%%%%%%%%%%%%%%%
\begin{frame}
\frametitle{Polysynthetische Sprachen}

\begin{itemize}
	\item auch \textbf{inkorporierende} Sprachen genannt
	\item[]
	\item syntaktische Beziehungen im Satz durch \textbf{Aneinanderreihen} und \textbf{Ineinanderfügen} lexikalischer und grammatischer Morpheme realisiert
	\item[]
	\item \zB wereden Subjekt- und Objektverhältnisse am Verb ausgedrückt
	\item[]
	\item Inuit, Irokesisch, Maya-Sprachen, Nahuatl, Sprachen im Pazifik-Raum wie Samoanisch, Tonga, Maori
	
	\ea
	\glll	ni kin ita k \\
			{Subj-1.Ps-Sing} {Obj.-3.Ps-Plur} seh Prät \\
			{nikinitak (Aztekisch (Zacapoaxatla))} \\
	\gq{ich sah sie (pl)}
	\z
	
\end{itemize}


\end{frame}


%%%%%%%%%%%%%%%%%%%%%%%%%%%%%%%%%
\begin{frame}
\frametitle{Polysynthetische Sprachen}

\begin{itemize}
	\item Auch im Deutschen kann man bestimmte Konstruktionen als Inkorporationen analysieren
	
	\begin{itemize}
		\item die Kombination Verb $+$ artikelloses Nomen
		\item das artikellose Nomen weist andere Eigenschaften auf als sein Gegenstück mit Artikel:
		
		\ea{ Andrea liest die Zeitung. \\
			 Andrea liest Zeitung.}
		\z
			 
		\ea {Andrea liest die informative Zeitung. \\
			 *Andrea liest informative Zeitung.}
		\z
			 
		\ea {Andrea liest eine Zeitung. Sie ist informativ. \\
			 Andrea liest Zeitung\textsubscript{i}. *Sie\textsubscript{i} ist informativ.}
		\z
			 
	\end{itemize}
	\item Mit anderen Worten: \\
		Die meisten Sprachen sind Mischformen der vier Typen.
\end{itemize}


\end{frame}


