%%%%%%%%%%%%%%%%%%%%%%%%%%%%%%%%%%
%% UE 3 - 05b Morphologie
%%%%%%%%%%%%%%%%%%%%%%%%%%%%%%%%%%

\begin{frame}
	\frametitle{Übung -- Lösung}
	
\settowidth\jamwidth{zwei Lesarten: Rektionskompositum und kein}
	
\begin{itemize}
	\item In welchen der folgenden Beispiele liegen Rektionskomposita vor?
		
	\begin{exe}
		\exr{ex:05bUE3.1}
		\begin{xlist}
			\ex Zigarrenraucher \loesung{2}{Rektionskompositum}
			\ex Gelegenheitsraucher \loesung{3}{kein Rektionskompositum}
			\ex Kettenraucher \loesung{4}{kein Rektionskompositum}
		\end{xlist}
		\exr{ex:05bUE3.2}
		\begin{xlist}
			\ex Hochschullehrer \loesung{5}{kein Rektionskompositum}
			\ex Mathematiklehrer \loesung{6}{Rektionskompositum}
		\end{xlist}
		\exr{ex:05bUE3.3}
		\begin{xlist}
			\ex hitzefrei \loesung{7}{kein Rektionskompositum}
			\ex kugelsicher \loesung{8}{Rektionskompositum}
		\end{xlist}
		\exr{ex:05bUE3.4}
		\begin{xlist}
			\ex WDR-Kritiker \loesung{9}{zwei Lesarten: Rektionskompositum}
			\loesung{9}{und kein Rektionskompositum}
		\end{xlist}
	\end{exe}

\end{itemize}
	
\end{frame}