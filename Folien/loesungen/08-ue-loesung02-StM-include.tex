%%%%%%%%%%%%%%%%%%%%%%%%%%%%%%%%%%
%% UE 2 - 08 Pragmatik stefan
%%%%%%%%%%%%%%%%%%%%%%%%%%%%%%%%%%

\begin{frame}
\frametitle{Übung -- Lösung}
	
	\begin{enumerate}
		\item Gegeben sei der Satz unter (\ref{ex:Prag42}):
		
	\begin{exe}
		\exr{ex:Prag42} Einige der US-amerikanischen Beamten wissen, wer Richard erdrosselt hat.
	\end{exe}

		\item [] Geben Sie bei jedem der Sätze unter (\ref{ex:Prag43})--(\ref{ex:Prag46}) an, ob es sich um eine Implikatur, oder ob es sich um eine Präsupposition zu (\ref{ex:08ue1}) handelt. Schreiben Sie die richtige Antwort hinter den jeweiligen Satz. Wenn es sich um eine Präsupposition handelt, testen Sie dies anhand eines der Präsuppositionstests. \\
		\textbf{NB:} Vorsicht, zuweilen wird keine der Relationen wiedergegeben!
	\begin{exe}
		\exr{ex:Prag43} Es existieren US-amerikanische Beamte. \alertgreen{\ras Präsupposition}
		\exr{ex:Prag44} Richard war ein Semantiker. \alertgreen{\ras keine Relation}
		\exr{ex:Prag45} Nicht alle US"=amerikanischen Beamten wissen, wer den Mord begangen hat. \alertgreen{\ras Implikatur}
		\exr{ex:Prag46} Richard wurde erdrosselt. \alertgreen{\ras Präsupposition}
	\end{exe}

	\end{enumerate}
	
\end{frame}

%%%%%%%%%%%%%%%%%%%%%%%%%%%%%%%%%%%%%%%%%%%

\begin{frame}
	\frametitle{Übung -- Lösung}
	
	\begin{enumerate}
		\item[2.] Bestimmen und kennzeichnen Sie zwei deiktische Ausdrücke im Satz (\ref{ex:Prag47}). Geben Sie zudem eine Anapher mit ihrem Antezendens an.
	
	\begin{exe}
		\exr{ex:Prag47} Angelika hat gestern erwähnt, dass Irene sich dort mit den Formeln amüsiert hat.
	\end{exe}
		
		\begin{itemize}
			\item[] \alertgreen{Ausdruck: \emph{gestern}, Art: Temporaldeixis}
			\item[] \alertgreen{Ausdruck: \emph{dort}, Art: Lokaldeixis}
			\item[] \alertgreen{Anapher: \emph{sich}, Antezedens: \emph{Irene}}
			\item[] \alertgreen{Auch möglich: \emph{den Formeln} \ras Objektdeixis}
		\end{itemize}
		
	\end{enumerate}
	
\end{frame}

%%%%%%%%%%%%%%%%%%%%%%%%%%%%%%%%%%%%%%%%%%%%

\begin{frame}
	\frametitle{Übung -- Lösung}
	
	\begin{enumerate}
		\item[3.] Kreuzen Sie für Satz (\ref{ex:Prag48}) alle Sätze in der unten stehenden Liste an, die (konversationelle) Implikaturen dieses Satzes darstellen.
		
	\begin{exe}
		\exr{ex:Prag48} Gottfried hat einige Nachbarn beleidigt.
	\end{exe}

		\begin{itemize}
			\item[$\circ$] Gottfried hat einen Nachbarn.
			\item[\alertgreen{$\checkmark$}] \alertgreen{Gottfried hat nicht alle Nachbarn beleidigt.}
			\item[$\circ$] Gottfried ist ein unbeliebter Mensch.
			\item[\alertgreen{$\checkmark$}] \alertgreen{Gottfried hat etwas Unhöfliches gesagt.}
			\item[\alertgreen{$\checkmark$}] \alertgreen{Gottfried hat einige Nachbarn nicht beleidigt.}
		\end{itemize}
		
	\end{enumerate}


\end{frame}