%%%%%%%%%%%%%%%%%%%%%%%%%%%%%%%%%%
%% HA 1 - 05c Morphologie
%%%%%%%%%%%%%%%%%%%%%%%%%%%%%%%%%%

\begin{frame}{Hausaufgabe -- Lösung}

\begin{enumerate}
	\item Kreuzen Sie die korrekten Aussagen an %\hfill(0,5 Punkte pro Aussage)\\

\begin{itemize}
	\item[$\circ$] Die Graphemkette abarbeiten ist ein einzelnes phonologisches Wort im Deutschen.
	\item[$\circ$] \emph{Morphologieeinführungsbuch} ist ein orthographisch-graphemisches Wort des Deutschen, sowie \emph{introductory morphology book} ein orthographisch-graphemisches Wort des Englischen ist.
	\item[$\circ$] Ein Morphem ist die kleinste bedeutungsunterscheidende Einheit in einem bestimmten Sprachsystem.
	\alertgreen{
			\item[\alertgreen{$\checkmark$}] \ab{Brot} und \ab{Bröt} sind Allomorphe eines einzelnen Morphems.
		}
\end{itemize}

\end{enumerate}
\end{frame}


%%%%%%%%%%%%%%%%%%%%%%%%%%%%%%%%%%
\begin{frame}{Hausaufgabe -- Lösung}

\begin{enumerate}
	\item[2.] Erklären Sie das Prinzip der Rechtsköpfigkeit in der Morphologie des Deutschen. Verwenden Sie bei Ihrer Erklärung die unten angegebenen Beispiele. %\hfill(4 Punkte)\\

\begin{exe}
	\exr{ex:05cHA2}
	\begin{xlist}
		\ex lichtblau, Blaulicht
		\ex die Fotowelt, das Weltfoto
		\ex	die Bücherrücken, die Rückenbücher
	\end{xlist}
\end{exe}

\pause

\alertgreen{Der Kopf eines Wortes ist immer rechtsperipher. Er bestimmt die morphosyntaktischen Eigenschaften eines Wortes sowie viele semantische Aspekte, \zB}

	\begin{itemize}
		\item[\alertgreen{--}] \alertgreen{(\ref{ex:05cHA2a}): Wortart (A \vs N)}
		\item[\alertgreen{--}] \alertgreen{(\ref{ex:05cHA2b}): Genus (f \vs n)}
		\item[\alertgreen{--}] \alertgreen{(\ref{ex:05cHA2c}): Pluralflexion (endungslos \vs \emph{-er})}
		\item[\alertgreen{--}] \alertgreen{Bei Determinativkomposita bildet das Kompositum eine Unterart des Kopfes, bspw. geht es in (\ref{ex:05cHA2c}) im ersten Fall um einen bestimmen Blauton und im zweiten um eine bestimmte Art von Licht.}
	\end{itemize}

\end{enumerate}
\end{frame}


%%%%%%%%%%%%%%%%%%%%%%%%%%%%%%%%%

\begin{frame}{Hausaufgabe -- Lösung}

\begin{enumerate}
\item[3.] Geben Sie Argumente für oder gegen die Behandlung von \emph{ver-} in den folgenden Wörtern als Morphem an. Wenn es sich um ein Morphem handelt, ist das immer das gleiche Morphem?%\\
%\hfill(4 Punkte)\\

\begin{exe}
	\exr{ex:05cHA3}
	\begin{xlist}
		\ex \emph{Ver}zweiflung
		\ex \emph{Ver}s
		\ex \emph{ver}kaufen
		\ex \emph{ver}schreiben
		\ex \emph{ver}fahren
	\end{xlist}
\end{exe}

\pause

\alertgreen{Morphem: Kleinste bedeutungstragende Einheit im Sprachsystem.}

\begin{itemize}
	\item[\alertgreen{--}] \alertgreen{\emph{ver-} in (\ref{ex:05cHA3b}) ist kein Morphem, sondern Bestandteil des Stammes.}
	\item[\alertgreen{--}] \alertgreen{\emph{ver-} in (\ref{ex:05cHA3d}) und (\ref{ex:05cHA3e}) ist ein Morphem mit der Bedeutung \gq{X falsch machen}.}
	\item[\alertgreen{--}] \alertgreen{\emph{ver-} in (\ref{ex:05cHA3a}) und (\ref{ex:05cHA3c}) sind auch Morpheme, aber zwei andere Morpheme, weil sie jeweils abweichende Bedeutungen tragen:}
		\begin{itemize}
			\item[] \alertgreen{\emph{ver-} in (\ref{ex:05cHA3c}) kehrt die Bedeutung von X um.}
			\item[] \alertgreen{\emph{ver-} in (\ref{ex:05cHA3a}) trägt eine intensivierende(?) Bedeutung.}
		\end{itemize}
\end{itemize}

\end{enumerate}
\end{frame}


%%%%%%%%%%%%%%%%%%%%%%%%%%%%%%%%%
\begin{frame}{Hausaufgabe -- Lösung}

\begin{enumerate}
\item[4.] Ordnen Sie die Wortbildungsprozesse links den passenden Beispielen rechts zu (dazu müssen Sie nur den entsprechenden Buchstaben neben das passende Beispiel schreiben). %\\
%\hfill(0,5 Punkte pro Aussage)\\

\begin{table}[h!]
	\begin{minipage}{0.4\linewidth}
		\centering
		\begin{tabular}{l|p{0.1\textwidth}|}
			Determinativkompositum & (A)\\
			\hline
			Konversion & (B)\\
			\hline
			Zirkumfigierung (Derivation) & (C)\\
			\hline
			Rektionskompositum & (D)\\
			\hline
			Possessivkompositum & (E)\\
		\end{tabular}
	
\end{minipage}\hfill%
\begin{minipage}{0.4\linewidth}
\centering
\scalebox{.9}{
		\begin{tabular}{|l|r}
			\only<2->{\alertgreen{C}} & \emph{Gerede} \\
			\hline
			\only<3->{\alertgreen{E}} & \emph{Milchgesicht}\\
			\hline
			\only<4->{\alertgreen{B}} & \emph{Lauf} \\
			\hline
			\only<5->{\alertgreen{A}} & \emph{Kettenraucher}  \\
			\hline
			\only<6->{\alertgreen{D}} & \emph{Klausurbesprechung}  \\
		\end{tabular}
}	
	\end{minipage}
\end{table}


\item[5.] Geben Sie für die folgende Wortform die Flexionskategorien an, nach denen sie flektiert ist. %\\
%\hfill(3 Punkte)\\
\end{enumerate}

\begin{columns}
\column[t]{.28\textwidth}
	\begin{exe}
		\exr{ex:05cHA5} bestehe
	\end{exe}

\column[t]{.6\textwidth}
\visible<7->{%
\alertgreen{%
1. / Sg. / Präsens / Indikativ / Aktiv\\
1. / Sg. / Präsens / Konjunktiv I / Aktiv\\
3. / Sg. / Präsens / Konjunktiv I / Aktiv\\
2. / Sg. / Präsens / Imperativ / Aktiv
}
}
\end{columns}



\end{frame}


%%%%%%%%%%%%%%%%%%%%%%%%%%%%%%%

\begin{frame}{Hausaufgabe -- Lösung}

\begin{enumerate}
	\item[6.] Warum sind die Wörter unter (\ref{ex:05cHA6a}) grammatisch und die unter (\ref{ex:05cHA6b}) ungrammatisch? %(4 Punkte)
	
	\begin{exe}
		\exr{ex:05cHA6}
		\begin{xlist}
			\ex kaufbar, trinkbar
			\ex *fensterbar, *helfbar, *schönbar
		\end{xlist}
	\end{exe}
	
\pause

\alertgreen{Das Suffix \emph{-bar} hat die folgenden Beschränkungen bzgl. der Basis X, mit der es sich verbindet:}
		\begin{itemize}
			\item[\alertgreen{--}] \alertgreen{X muss ein Verb sein (nicht Nomen oder Adjektiv)}
			\item[\alertgreen{--}] \alertgreen{X muss transitiv sein (nicht wie \emph{helfen})}
		\end{itemize}

\end{enumerate}
\end{frame}


%%%%%%%%%%%%%%%%%%%%%%%%%%%%%%%

\begin{frame}{Hausaufgabe -- Lösung}

\begin{enumerate}
\item[7.] Sind die folgenden Verben Präfixverben oder Partikelverben? Begründen Sie Ihre Entscheidungen. %\hfill(3 Punkte)\\

\begin{exe}
	\exr{ex:05cHA7}
	\begin{xlist}
		\ex auskennen
		\ex erkennen
		\ex aberkennen
	\end{xlist}
\end{exe}

\pause

\alertgreen{Partikelverben sind:}
\begin{itemize}
	\item[\alertgreen{--}] \alertgreen{morphologisch trennbar (\emph{aus-ge-kannt}, \emph{ab-zu-erkennen}). }
	\item[\alertgreen{--}] \alertgreen{syntaktisch trennbar (\emph{Peter kennt sich aus.}, \emph{Die Frau erkennt die Urkunde ab.}). }
	\item[\alertgreen{--}] \alertgreen{betont (\emph{\textprimstress auskennen} und \emph{\textprimstress aberkennen}).}
\end{itemize}
		
\pause
\medskip
		
\alertgreen{Präfixverben sind:}
\begin{itemize}
	\item[\alertgreen{--}] \alertgreen{weder morphologisch noch syntaktisch trennbar (*\emph{ergekannt}, *\emph{Sie kannte ihn er.}). }
	\item[\alertgreen{--}] \alertgreen{nicht betont (\emph{er}\textprimstress \emph{kennen}). }
\end{itemize}

\pause

\alertgreen{\emph{aberkennen} ist ein Partikelverb, welches aus einem Präfixverb und einer Partikel besteht (ab$+$erkennen).}

\end{enumerate}
\end{frame}


%%%%%%%%%%%%%%%%%%%%%%%%%%%%%%%%%

\begin{frame}{Hausaufgabe -- Lösung}

\begin{enumerate}
\item[8.] Geben Sie für das folgende Wort eine morphologische Konstituentenstruktur (inklusive Konstituentenkategorien (N, N\textsuperscript{af}, V, V\textsuperscript{af}, \dots)) an, und bestimmen Sie für jeden Knoten den Wortbildungstyp. %\hfill(6,5 Punkte)\\

\begin{exe}
	\exr{ex:05cHA8} Wahlkampfberaterinnen
\end{exe}

\end{enumerate}

\vspace{-.25cm}

\begin{figure}
\centering

\scalebox{.6}{

\alertgreen{
\begin{forest} MyP edges,
	[N, name=N1
	[N, name=N2
	[N, name=N3
	[N, name=N4
	[N, name=N6 [V[wahl/wähl]]]
	[N, name=N7 [V[kampf/kämpf]]]]
	[N, name=N5[V, name=V1	[V\textsubscript{af}[be-]]
	[V[rat]]]
	[N\textsuperscript{af}[-er]]]]
	[N\textsuperscript{af}[-in]]]
	[Fl[-nen]]]	
{
	\draw[<-, HUgreen] (N1.west)--++(-12em,0pt)
	node[anchor=east,align=center]{Flexion (KEIN Wortbildungsporzess)};
	\draw[<-, HUgreen] (N2.west)--++(-14em,0pt)
	node[anchor=east,align=center]{Derivation (Movierung)};
	\draw[<-, HUgreen] (N3.west)--++(-9.5em,0pt)
	node[anchor=east,align=center]{Determinativkompositum};
	\draw[<-, HUgreen] (N4.west)--++(-4em,0pt)
	node[anchor=east,align=center]{Determinativkompositum};
	\draw[<-, HUgreen] (N5.west)--++(-3em,0pt)
	node[anchor=east,align=center]{Derivation};
	\draw[<-, HUgreen] (N6.west)--++(-3em,0pt)
	node[anchor=east,align=center]{Implizite Derivation};
	\draw[<-, HUgreen] (N7.east)--++(2.5em,0pt)--++(0em,-18ex)%--++(2em,0pt)
	node[anchor=north,align=center]{Implizite Derivation};
	\draw[<-, HUgreen] (V1.east)--++(1.5em,0pt)--++(0em,-14ex)--++(2em,0pt)
	node[anchor=west,align=center]{Derivation};
}
\end{forest}	
%\begin{itemize}
%	\item[]N1: Flexion (KEIN Wortbildungsporzess)
%	\item[]N2: Derivation (Movierung)
%	\item[]N3: Determinativkompositum
%	\item[]N4: Determinativkompositum
%	\item[]N5: Derivation
%	\item[]N6: Implizite Derivation
%	\item[]N7: Implizite Derivation
%	\item[]V1: Derivation
%\end{itemize}
}
}

\end{figure}
\end{frame}


%%%%%%%%%%%%%%%%%%%%%%%%%%%%%%%%%%%

\begin{frame}{Hausaufgabe -- Lösung}

\begin{enumerate}
\item[9.] Paraphrasieren Sie das folgende komplexe Wort so, dass es der angegebenen Struktur entspricht (auch wenn Sie selbst eine andere Struktur plausibler finden sollten). %\\
%\hfill(2 Punkte)\\


\begin{forest}sn edges,
	[N
	[N[N[Reserve]]
	[N[V[lehr]][N\textsuperscript{af}[-er]]]]
	[N[zimmer]]
	]
\end{forest}

\pause

\alertgreen{Ein Zimmer für Reservelehrer}

\end{enumerate}
\end{frame}

