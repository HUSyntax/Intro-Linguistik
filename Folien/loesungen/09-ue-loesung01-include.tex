%%%%%%%%%%%%%%%%%%%%%%%%%%%%%%%%%%
%% UE 1 - 09 Übungen
%%%%%%%%%%%%%%%%%%%%%%%%%%%%%%%%%%

\begin{frame}
\frametitle{Übung: Phonetik/Phonologie -- Lösung}

\begin{itemize}
	\item[1.] Erläutern Sie den Unterschied zwischen Phon, Phonem und Allophon.
	
	\item Phon:
		
		\only<2->{
		\begin{itemize}
			\item \alertgreen{Minimaleinheit der Phonetik}
			\item \alertgreen{physikalisch messbare lautliche Einheit einer Sprache} \pause
		\end{itemize}
	}
		
	\item Phonem:
		
		\only<3->{
		\begin{itemize}
			\item \alertgreen{Minimaleinheit der Phonologie}
			\item \alertgreen{abstraktes Konstrukt, steht für eine Menge von möglichen Phonen}
			\item \alertgreen{ermittelbar durch Minimalpaarbildung (strukturalistisches Kriterium)} \pause
		\end{itemize}
	}
		
	\item Allophon:
		
		\only<4->{
		\begin{itemize}
			\item \alertgreen{phonetische Realisierungsvariante eines Phonems}
			\item \alertgreen{Untertypen: komplementäre und freie Allophonie, regionale und soziale Variation}
		\end{itemize}
		}

\end{itemize}

\end{frame}

%%%%%%%%%%%%%%%%%%%%%%%%%%%%%%%%%%%

\begin{frame}
	
\begin{itemize}
	\item[2.] Geben sie die artikulatorischen Eigenschaften der folgenden Laute an.
	
	\begin{exe}
		\exr{ex:01}
	\settowidth \jamwidth{\alertgreen{halbhoher fast vorderer ungerundeter ungespannter Vokal}}
	
		\begin{xlist}
			\ex \textipa{[r]}	\only<2->{\jambox{\alertgreen{alveolarer stimmhafter Vibrant}}}
			\ex \textipa{[P]}	\only<3->{\jambox{\alertgreen{glottaler stimmloser Plosiv}}}
			\ex \textipa{[b]}	\only<4->{\jambox{\alertgreen{bilabialer stimmhafter Plosiv}}}
			\ex \textipa{[f]}	 \only<5->{\jambox{\alertgreen{labiodentaler stimmloser Frikativ}}}
			\ex \textipa{[I]}	 \only<6->{\jambox{\alertgreen{halbhoher fast vorderer ungerundeter ungespannter Vokal}}}
			\ex \textipa{[u:]}	\only<7->{\jambox{\alertgreen{hoher hinterer gerundeter gespannter (langer) Vokal}}}
		\end{xlist}
	
	\end{exe}

\end{itemize}

\end{frame}

%%%%%%%%%%%%%%%%%%%%%%%%%%%%%%%%%%%

\begin{frame}
	
\begin{itemize}
	\item[3.] Geben Sie die phonologische Repräsentation und die phonetische standarddeutsche Transkription der folgenden Wörter mit Silbenstruktur und X-Skelettschicht an.
		
	\begin{exe}
		\exr{ex:02}
		
		\begin{xlist}
			\ex Näherinnen
			\ex Zwischendinger
			\ex königlich
		\end{xlist}
	
	\end{exe}

\end{itemize}

\end{frame}

%%%%%%%%%%%%%%%%%%%%%%%%%%%%%%%%%%%	
	
\begin{frame}

(2a): Näherinnen \\

\medskip

	\alertgreen{\textipa{/ne:.@.KI\d{n}@n/} \ras \textipa{[\textprimstress ne:.@.KI\d{n}@n]} }
	
	\centering
	\alertgreen{
	\scalebox{1}{
	\begin{forest} MyP edges, [,phantom
		[$\sigma$
		[O
			[x, tier=word[\textipa{n}]]
		]
		[R
			[N
				[x, tier=word[\textipa{E:}]]
			]
			[K]
		]
		]
		[$\sigma$
		[O]
		[R
			[N
				[x, tier=word[\textipa{@}]]
			]
			[K]
		]
		]
		[$\sigma$
		[O
			[x, tier=word[\textipa{K}]]+
		]
		[R
			[N
				[x, tier=word[\textipa{I}]]
			]
			[K
				[x, tier=word, name=x[\textipa{n}]]
			]
		]
		]
		[$\sigma$
		[O, name=O]
		[R
			[N
				[x, tier=word[\textipa{@}]]
			]
			[K
				[x, tier=word[\textipa{n}]]
			]
		]
		]
	]
	\draw[HUgreen] (O.south)--(x.north);
	\end{forest}
	}}

\end{frame}

%%%%%%%%%%%%%%%%%%%%%%%%%%%%%%%%%%%

\begin{frame}

(2b): Zwischendinger	\\

\medskip

	\alertgreen{\textipa{/\t{ts}vI\d{S}@n.dIn.g@\textscr/} \ras \textipa{[\textprimstress \t{ts}vI\d{S}@n.dI\d{N}5]}}
	
	\centering
	\alertgreen{
	\scalebox{1}{
	\begin{forest} MyP edges, [,phantom
		[$\sigma$
		[O
			[x, tier=word[\textipa{\t{ts}}] ]
			[x, tier=word[\textipa{v}] ]
		]
		[R
			[N
				[x, tier=word[\textipa{I}]]
			]
			[K
				[x, tier=word, name=x[\textipa{S}]]
			]
		]
		]
		[$\sigma$
		[O, name=O]
		[R
			[N
				[x, tier=word[\textipa{@}] ]
			]
			[K
				[x, tier=word[\textipa{n}]]
			]
		]
		]
		[$\sigma$
		[O
			[x, tier=word[\textipa{d}]]
		]
		[R
			[N
				[x, tier=word[\textipa{I}]]
			]
			[K
				[x, tier=word, name=x2[\textipa{N}]]
			]
		]
		]
		[$\sigma$
		[O, name=O2]
		[R
			[N
				[x, tier=word[\textipa{5}]]
			]
			[K]	
		]
		]
	]
	\draw[HUgreen] (O.south)--(x.north);
	\draw[HUgreen] (O2.south)--(x2.north);
	\end{forest}
	} }

\end{frame}

%%%%%%%%%%%%%%%%%%%%%%%%%%%%%%%%%%%

\begin{frame}

(2c): königlich \\

\medskip

	\alertgreen{\textipa{/k\o:.nIg.lI\c{c}/} \ras \textipa{[\textprimstress k\o:.nIk.lI\c{c}]}} ~\\
	
	\centering
	\alertgreen{
	\scalebox{1}{
	\begin{forest} MyP edges, [,phantom
		[$\sigma$
		[O
			[x, tier=word[\textipa{k}]]
		]
		[R
			[N
				[x, tier=word[\textipa{\o:}, name=o]]
				[x, tier=word, name=x]		
			]
			[K]
		]
		]
		[$\sigma$
		[O
			[x, tier=word[\textipa{n}]]
		]
		[R
			[N
				[x, tier=word[\textipa{I}]]
			]
			[K
				[x, tier=word[\textipa{k}]]
			]
		]
		]
		[$\sigma$
		[O
			[x, tier=word[\textipa{l}]]
		]
		[R
			[N
				[x, tier=word[\textipa{I}]]
			]
			[K
				[x, tier=word[\textipa{\c{c}}]]
			]
		]
		]
	]
	\draw[HUgreen] (x.south)--(o.north);		
	\end{forest}
	} }

\end{frame}

%%%%%%%%%%%%%%%%%%%%%%%%%%%%%%%%%%%
\begin{frame}

\begin{itemize}
	\item[4.] Benennen Sie die phonetisch/phonologischen Prozesse, die stattfinden, bei der Aussprache der folgenden Wörter:
	
	\begin{exe}
		\exr{ex:03}
	\settowidth \jamwidth{\alertgreen{regressive velare Nasalassimilation: \textipa{/n/} \ras \textipa{[N]}},}
	
		\begin{xlist}
			\ex mild	\only<2->{\jambox{\alertgreen{Auslautverhärtung: \textipa{/d/} \ras \textipa{[t]}}}}
\medskip
			\ex ungelenkig	\only<3->{\jambox{\alertgreen{Knacklauteinsetzung: $\emptyset$ \ras \textipa{[P]}},}}
			\only<3->{\jambox{\alertgreen{regressive velare Nasalassimilation: \textipa{/n/} \ras \textipa{[N]}},}} 
			\only<3->{\jambox{\alertgreen{g-Spirantisierung: \textipa{/g/} \ras \textipa{[\c{c}]}}}}
\medskip
			\ex süchtig	\only<4->{\jambox{\alertgreen{g-Spirantisierung: \textipa{/g/} \ras \textipa{[\c{c}]}}}}
\medskip
			\ex Kraken	\only<5->{\jambox{\alertgreen{Schwa-Tilgung: \textipa{/@/} \ras $\emptyset$},}}
			\only<5->{\jambox{\alertgreen{progressive Ortsassimilation: \textipa{/kn/} \ras \textipa{[kN]}}}}
		\end{xlist}
	
	\end{exe}

\end{itemize}

\end{frame}

%%%%%%%%%%%%%%%%%%%%%%%%%%%%%%%%%%%
\begin{frame}

\begin{itemize}
	\item[5.] Sind die folgenden Segmentfolgen mögliche phonetische Wörter des Standarddeutschen?
	
	\begin{exe}
		\exr{ex:04} \textipa{[p@:kl.\textprimstress Ipl]}
		\exr{ex:05} \textipa{[\textprimstress Na:h.i:ltd]}
	
	\end{exe}
	
	\only<2->{\item \alertgreen{Beispiel (\ref{ex:04}) kann kein phonetisches Wort des Standarddeutschen sein, denn: gespanntes \textipa{[@]}, Verletzung der Sonoritätshierarchie in der Koda oder Onset-Maximierung in der folgenden Silbe \textipa{[kl]}, keine Knacklauteinsetzung in betonter Silbe, Verletzung der Sonoritätshierarchie \textipa{[pl]}}}
	
	\only<3->{\item \alertgreen{Beispiel (\ref{ex:05}) kann kein phonetisches Wort des Standarddeutschen sein, denn: \textipa{[N]} am Wortanfang, \textipa{[h]} wird wortintern nicht realisiert, \textipa{[i:]} muss mit daraffolgendem Konsonanten kurz sein, fehlende Auslautverhärtung \textipa{[d]}, anschließende fehlende Geminatenreduktion \textipa{[tt]} }}
\end{itemize}

\end{frame}
%%%%%%%%%%%%%%%%%%%%%%%%%%%%%%%%%%%