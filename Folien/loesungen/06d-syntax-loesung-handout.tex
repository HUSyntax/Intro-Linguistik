%%%%%%%%%%%%%%%%%%%%%%%%%%%%%%%%%%%%%%%%%%%%%%
%% Compile: XeLaTeX BibTeX XeLaTeX XeLaTeX
%% Loesung-Handout: Antonio Machicao y Priemer
%% Course: GK Linguistik
%%%%%%%%%%%%%%%%%%%%%%%%%%%%%%%%%%%%%%%%%%%%%%

%\documentclass[a4paper,10pt, bibtotoc]{beamer}
\documentclass[10pt,handout]{beamer}

%%%%%%%%%%%%%%%%%%%%%%%%
%%     PACKAGES      
%%%%%%%%%%%%%%%%%%%%%%%%

%%%%%%%%%%%%%%%%%%%%%%%%
%%     PACKAGES       %%
%%%%%%%%%%%%%%%%%%%%%%%%



%\usepackage[utf8]{inputenc}
%\usepackage[vietnamese, english,ngerman]{babel}   % seems incompatible with german.sty
%\usepackage[T3,T1]{fontenc} breaks xelatex

\usepackage{lmodern}
\usepackage{calligra}

\usepackage{amsmath}
\usepackage{amsfonts}
\usepackage{amssymb}
%% MnSymbol: Mathematische Klammern und Symbole (Inkompatibel mit ams-Packages!)
%% Bedeutungs- und Graphemklammern: $\lsem$ Tisch $\rsem$ $\langle TEXT \rangle$ $\llangle$ TEXT $\rrangle$ 
\usepackage{MnSymbol}
%% ulem: Strike out
\usepackage[normalem]{ulem}  

%% Special Spaces (s. Commands)
\usepackage{xspace}				
\usepackage{setspace}
%	\onehalfspacing

%% mdwlist: Special lists
\usepackage{mdwlist}	


%%%%%%%%%%%%%%%%%%%%%%%%%%%%%%%%
%% TIPA & Phonetics

\usepackage[
%noenc,
safe]{tipa}

%% TIPA Problems/Solutions:
%% Problems with U, serif fonts and ligatures

%%Test 1
%\DeclareFontSubstitution{T3}{cmss}{m}{n}

%%Test 2
%\DeclareFontSubstitution{T3}{ptm}{m}{n}

%%Test 3
%\usepackage{tipx}


%\usepackage{vowel}


%%%%%%%%%%%%%%%%%%%%%%%%%%%%%%%%
%% Examples

\usepackage{jambox}



%\usepackage{forest-v105}
%\usepackage{langsci-forest-v105-setup}


%%%%%%%%%%%%%%%%%%%%%%%%%%%%%%%%
%% Fonts for Chinese, Vietnamese, etc. (s. Graphematik)

\usepackage{xeCJK}
\setCJKmainfont{SimSun}


%\usepackage{natbib}
%\setcitestyle{notesep={:~}}


% for toggles, is loaded in hu-beamer-includes-pdflatex
%\usepackage{etex}


%%%%%%%%%%%%%%%%%%%%%%%%%%%%%%%%
%% Fonts for Fraktur

\usepackage{yfonts}

\usepackage{url}

% für UDOP
\usepackage{adjustbox}


%% huberlin: Style sheet
%\usepackage{huberlin}
\usepackage{hu-beamer-includes-pdflatex}
\huberlinlogon{0.86cm}

% %% % use this definition, if you want to see the outlines in the handout
\renewcommand{\outline}[1]{%
%\beamertemplateemptyfootbar%
\huberlinjustbarfootline
\frame{\frametitle{\outlineheading}#1}%
%\beamertemplatecopyrightfootframenumber%
\huberlinnormalfootline 
\huberlinpagedec
}



%% Last Packages
%\usepackage{hyperref}	%URLs
%\usepackage{gb4e}		%Linguistic examples

% sorry this was incompatible with gb4e and had to go.
%\usepackage{linguex-cgloss}	%Linguistic examples (patched version that works with jambox

\usepackage{multirow}  %Mehrere Zeilen in einer Tabelle
\usepackage{adjustbox} %adjusting tables
%\usepackage{array}
\usepackage{marginnote}	%Notizen




%%%%%%%%%%%%%%%%%%%%%%%%%%%%%%%%%%%%%%%%%%%%%%%%%%%%
%%%            MyP-Commands                     
%%%%%%%%%%%%%%%%%%%%%%%%%%%%%%%%%%%%%%%%%%%%%%%%%%%%


%%%%%%%%%%%%%%%%%%%%%%%%%%%%%%%%
% Delete Caption from Figures and Tables
\setbeamertemplate{caption}{\centering\insertcaption\par }


%%%%%%%%%%%%%%%%%%%%%%%%%%%%%%%%
% German quotation marks:
\newcommand{\gqq}[1]{\glqq{}#1\grqq{}}		%double
\newcommand{\gq}[1]{\glq{}#1\grq{}}			%simple


%%%%%%%%%%%%%%%%%%%%%%%%%%%%%%%%
% Abbreviations in German
% package needed: xspace
% Short space in German abbreviations: \,	
\newcommand{\idR}{\mbox{i.\,d.\,R.}\xspace}
\newcommand{\su}{\mbox{s.\,u.}\xspace}
%\newcommand{\ua}{\mbox{u.\,a.}\xspace}       % in abbrev
\newcommand{\vgl}{\mbox{vgl.}\xspace}       % in abbrev
%\newcommand{\zB}{\mbox{z.\,B.}\xspace}       % in abbrev
%\newcommand{\s}{s.~}
%not possibel: \dh --> d.\,h.

%rot unterstrichen
%\newcommand{\rotul}[1]{\textcolor{red}{\underline{#1}}}

%%%%%%%%%%%%%%%%%%%%%%%%%%%%%%%%
%Abbreviations in English
\newcommand{\ao}{a.o.\ }	% among others
%\newcommand{\cf}[1]{(cf.~#1)}	% confer = compare
\renewcommand{\ia}{i.a.}	% inter alia = among others
%\newcommand{\ie}{i.e.~}	% id est = that is
\newcommand{\fe}{e.g.~}	% exempli gratia = for example
%not possible: \eg --> e.g.~
\newcommand{\vs}{vs.\ }	% versus
\newcommand{\wrt}{w.r.t.\ }	% with respect to


%%%%%%%%%%%%%%%%%%%%%%%%%%%%%%%%
% Dash:
\newcommand{\gs}[1]{--\,#1\,--}


%%%%%%%%%%%%%%%%%%%%%%%%%%%%%%%%
% Rightarrow with and without space
\def\ra{\ensuremath\rightarrow}			%without space
\def\ras{\ensuremath\rightarrow\ }		%with space


%%%%%%%%%%%%%%%%%%%%%%%%%%%%%%%%
%% X-bar notation

%% Notation with primes (not emphasized): \xbar{X}
\newcommand{\MyPxbar}[1]{#1$^{\prime}$}
\newcommand{\xxbar}[1]{#1$^{\prime\prime}$}
\newcommand{\xxxbar}[1]{#1$^{\prime\prime\prime}$}

%% Notation with primes (emphasized): \exbar{X}
\newcommand{\exbar}[1]{\emph{#1}$^{\prime}$}
\newcommand{\exxbar}[1]{\emph{#1}$^{\prime\prime}$}
\newcommand{\exxxbar}[1]{\emph{#1}$^{\prime\prime\prime}$}

% Notation with zero and max (not emphasized): \xbar{X}
\newcommand{\zerobar}[1]{#1$^{0}$}
\newcommand{\maxbar}[1]{#1$^{\textsc{max}}$}

% Notation with zero and max (emphasized): \xbar{X}
\newcommand{\ezerobar}[1]{\emph{#1}$^{0}$}
\newcommand{\emaxbar}[1]{\emph{#1}$^{\textsc{max}}$}

%% Notation with bars (already implemented in gb4e):
% \obar{X}, \ibar{X}, \iibar{X}, \mbar{X} %Problems with \mbar!
%
%% Without gb4e:
\newcommand{\overbar}[1]{\mkern 1.5mu\overline{\mkern-1.5mu#1\mkern-1.5mu}\mkern 1.5mu}
%
%% OR:
\newcommand{\MyPibar}[1]{$\overline{\textrm{#1}}$}
\newcommand{\MyPiibar}[1]{$\overline{\overline{\textrm{#1}}}$}
%% (emphasized):
\newcommand{\eibar}[1]{$\overline{#1}$}
\newcommand{\eiibar}[1]{\overline{$\overline{#1}}$}

%%%%%%%%%%%%%%%%%%%%%%%%%%%%%%%%
%% Subscript & Superscript: no italics
\newcommand{\MyPdown}[1]{\textsubscript{#1}}
\newcommand{\MyPup}[1]{\textsuperscript{#1}}

%%%%%%%%%%%%%%%%%%%%%%%%%%%%%%%%
%% Small caps subscripts & superscripts
\newcommand{\scdown}[1]{\textsubscript{\textsc{#1}}}
\newcommand{\scup}[1]{\textsuperscript{\textsc{#1}}}

%%%%%%%%%%%%%%%%%%%%%%%%%%%%%%%%
% Objekt language marking:
%\newcommand{\obj}[1]{\glqq{}#1\grqq{}}	%German double quotes
%\newcommand{\obj}[1]{``#1''}			%English double quotes
%\newcommand{\MyPobj}[1]{\emph{#1}}		%Emphasising
\newcommand{\MyPobj}[1]{\textit{#1}}		%Emphasising

%%%%%%%%%%%%%%%%%%%%%%%%%%%%%%%%
% Size:
\newcommand{\size}[1]{#1}	% f.e. resize citations


%%%%%%%%%%%%%%%%%%%%%%%%%%%%%%%%
%% Semantic types (<e,t>), features, variables and graphemes in angled brackets 

%%% types and variables, in math mode: angled brackets + italics + no space
%\newcommand{\type}[1]{$<#1>$}

%%% OR more correctly: 
%%% types and variables, in math mode: chevrons! + italics + no space
\newcommand{\MyPtype}[1]{$\langle #1 \rangle$}

%%% features and graphemes, in math mode: chevrons! + italics + no space
\newcommand{\abe}[1]{$\langle #1 \rangle$}


%%% features and graphemes, in math mode: chevrons! + no italics + space
\newcommand{\ab}[1]{$\langle$#1$\rangle$}  %%same as \abu  
\newcommand{\abu}[1]{$\langle$#1$\rangle$} %%Umlaute


%% Presuppositions
\newcommand{\prspp}{$\gg$} 

%% Implicature
\newcommand{\implc}{$+ \mkern-5mu >$} 

%% Enttailment
\newcommand{\ent}{$\vDash$}

%% Other semantic symbols: 
%% entailment: $\Rightarrow$ $\vDash$
%% equivalence: $\Leftrightarrow$ $\equiv$
%% biconditional: $\leftrightarrow$ 
%% lexical rule: $\mapsto$
%% greater/less/equal: $>$ $\geq$ $<$ $\leq$
%% definition: $:=$ $=$\textsubscript{def}


%%%%%%%%%%%%%%%%%%%%%%%%%%%%%%%%
% Marking text with colour:
% package needed: xcolor
% Command \alert{} in Beamer >> FU-grün (leider!! @Stefan)

%%%neue Farbbefehle in Anlehnung an rotul
%%%(s. hu-beamer-includes-pdflatex.sty in texmf)

%% Farbdefinitionen:

\definecolor{HUred}{RGB}{138,15,20}
\definecolor{HUblue}{RGB}{0,55,108}
\definecolor{HUgreen}{RGB}{0,87,44}

%\newcommand{\alertred}[1]{\textcolor{red}{#1}}  % basic red
\newcommand<>{\alertred}[1]{{\color#2[RGB]{138,15,20}#1}}  %HU rot + overlay

%\newcommand{\alertblue}[1]{\textcolor{blue}{#1}} 		% basic blue
\newcommand<>{\alertblue}[1]{{\color#2[RGB]{0,55,108}#1}} %HU blue + overlay

%\newcommand{\alertgreen}[1]{\textcolor{green}{#1}}	% basic green
\newcommand<>{\alertgreen}[1]{{\color#2[RGB]{0,87,44}#1}} %HU green + overlay


%%% Verwendung der oben definierten Farben mit Unterschied in Handout und Beamer:

\mode<handout>{%
	\newcommand<>{\hured}[1]{\only#2{\underline{#1}}}
	\newcommand<>{\hublue}[1]{\only#2{\textbf{#1}}}
	\newcommand<>{\hugreen}[1]{\only#2{\textsc{#1}}}
}
%
\mode<beamer>{%
	\newcommand<>{\hured}[1]{\alertred#2{#1}}
	\newcommand<>{\hublue}[1]{\alertblue#2{#1}}
	\newcommand<>{\hugreen}[1]{\alertgreen#2{#1}}
}


%%%%%%%%%%%%%%%%%%%%%%%%%%%%%%%%
%% Outputbox
\newcommand{\outputbox}[1]{\noindent\fbox{\parbox[t][][t]{0.98\linewidth}{#1}}\vspace{0.5em}}


%%%%%%%%%%%%%%%%%%%%%%%%%%%%%%%%
%% (Syntactic) Trees
% package needed: forest
%
%% Setting for simple trees
\forestset{
	MyP edges/.style={for tree={parent anchor=south, child anchor=north}}
}

%% this is taken from langsci-setup file
%% Setting for complex trees
%% \forestset{
%% 	sm edges/.style={for tree={parent anchor=south, child anchor=north,align=center}}, 
%% background tree/.style={for tree={text opacity=0.2,draw opacity=0.2,edge={draw opacity=0.2}}}
%% }

\newcommand\HideWd[1]{%
	\makebox[0pt]{#1}%
}

%%%%%%%%%%%%%%%%%%%%%%%%%%%%%%%
%%solutions in green + w/ jambox
\newcommand{\loesung}[2]{\jambox{\visible<#1->{\alertgreen{#2}}}}

%%%%%%%%%%%%%%%%%%%%%%%%%%%%%%%%
%% TIPA Lösungen           

%%Tipa serif font fixed (requires package 'Linux Libertine B')

%% Solution 1 (RF)
%% Tipa font:
%\renewcommand\textipa[1]{{\fontfamily{cmr}\tipaencoding #1}}

%% Solution 2 (RF): older code for texlive 2017?
%\newfontfamily{\tipacm}[Scale=MatchUppercase]{Linux Libertine B}
%\renewcommand\useTIPAfont{\tipacm}

%\NewEnviron{IPA}{\expandafter\textipa\expandafter{\BODY}} %% not needed anymore

%% Solution 3 (RF): this solution is working but with problems with ligatures
%%% works for texlive 2018
\newfontfamily{\ipafont}[Scale=MatchUppercase]{Linux Libertine B}
\def\useTIPAfont{\ipafont}

%% Solution 4 (Kopecky & MyP): Test package: tipx (s. localpackages) and comment "Solution 3" 


%%%%%%%%%%%%%%%%%%%%%%%%%%%%%%%%
%% Toggles                  


\newtoggle{uebung}
\newtoggle{loesung}\togglefalse{loesung}

\newtoggle{hausaufgabe}

%\newtoggle{ha-loesung}\togglefalse{ha-loesung}
\newtoggle{phonologie-loesung}
\newtoggle{graphematik-loesung}


%% Neue Toggle-Struktur
\newtoggle{toc}
\newtoggle{sectoc}
\newtoggle{gliederung}

\newtoggle{ue-loesung}
\newtoggle{ha-loesung}
%%

% The toc is not needed on Handouts. Save trees.
\mode<handout>{
\togglefalse{toc}
}

\newtoggle{hpsgvorlesung}\togglefalse{hpsgvorlesung}
\newtoggle{syntaxvorlesungen}\togglefalse{syntaxvorlesungen}

%\includecomment{psgbegriffe}
%\excludecomment{konstituentenprobleme}
%\includecomment{konstituentenprobleme-hinweis}

\newtoggle{konstituentenprobleme}\togglefalse{konstituentenprobleme}
\newtoggle{konstituentenprobleme-hinweis}\toggletrue{konstituentenprobleme-hinweis}

%\includecomment{einfsprachwiss-include}
%\excludecomment{einfsprachwiss-exclude}
\newtoggle{einfsprachwiss-include}\toggletrue{einfsprachwiss-include}
\newtoggle{einfsprachwiss-exclude}\togglefalse{einfsprachwiss-exclude}

\newtoggle{psgbegriffe}\toggletrue{psgbegriffe}

\newtoggle{gb-intro}\togglefalse{gb-intro}


%%%%%%%%%%%%%%%%%%%%%%%%%%%%%%%%
%% Useful commands                    

%%%%%%%%%%%%%%%%%%%%%
%% FOR ITEMS:
%\begin{itemize}
%  \item<2-> from point 2
%  \item<3-> from point 3 
%  \item<4-> from point 4 
%\end{itemize}
%
% or: \onslide<2->
% or \only<2->{Text}
% or: \pause

%%%%%%%%%%%%%%%%%%%%%
%% VERTICAL SPACE:
% \vspace{.5cm}
% \vfill

%%%%%%%%%%%%%%%%%%%%%
% RED MARKING OF TEXT:
%\alert{bis spätestens Mittwoch, 18 Uhr}
%\newcommand{\alertred}[1]{\textcolor{red}{#1}}

%%%%%%%%%%%%%%%%%%%%%
%% RESCALE BIG TABLES:
%\scalebox{0.8}{
%For Big Tables
%}

%%%%%%%%%%%%%%%%%%%%%
%% BLOCKS:
%\begin{alertblock}{Title}
%Text
%\end{alertblock}
%
%\begin{block}{Title}
%Text
%\end{block}
%
%\begin{exampleblock}{Title}
%Text
%\end{exampleblock}

%%%%%%%%%%%%%%%%%%%%%
%% JAMBOX FOR EXAMPLES:
%\ea 
%\settowidth\jamwidth{Test} 
%Die Studierenden, die weitgehend von Stipendien leben, erhalten einen Mietzuschuss. 
%\jambox{Test}
%\z 

%%%%%%%%%%%%%%%%%%%%%
%% TOGGLES:


%%%%%%%%%%%%%%%%%%%%%%%%%%%%%%%%%%
%%%%%%%%%%%%%%%%%%%%%%%%%%%%%%%%%%
%\subsection{Übung}
%
%%%%%%%%%%%%%%%%%%%%%%%%%%%%%%%%%%
%%%%%%%%%%%%%%%%%%%%%%%%%%%%%%%%%%
%\iftoggle{uebung}{
%%%%%%%%%%%%%%%%%%%%%%%%%%%%%%%%%%
%\begin{frame}
%\frametitle{Übung}
%
%\end{frame}
%
%} 
%%% END true = Q
%%% BEGIN false = Q + A
%{
%%%%%%%%%%%%%%%%%%%%%%%%%%%%%%%%%%
%\begin{frame}
%\frametitle{Übung}
%
%\end{frame}
%%%%%%%%%%%%%%%%%%%%%%%%%%%%%%%%%%
%
%\begin{frame}
%\frametitle{Lösung}
%
%\end{frame}
%
%}%% END LOESUNG	
%%%%%%%%%%%%%%%%%%%%%%%%%%%%%%%%%%


%%%%%%%%%%%%%%%%%%%%%%%%%%%%%%%%%%
%%%%%%%%%%%%%%%%%%%%%%%%%%%%%%%%%%
%\subsection{Hausaufgabe}
%
%%%%%%%%%%%%%%%%%%%%%%%%%%%%%%%%%%
%%%%%%%%%%%%%%%%%%%%%%%%%%%%%%%%%%
%\iftoggle{hausaufgabe}{
%%%%%%%%%%%%%%%%%%%%%%%%%%%%%%%%%%
%
%\begin{frame}
%\frametitle{Hausaufgabe}
%
%\end{frame}
%
%} 
%%% END true = Q
%%% BEGIN false = Q + A
%{
%%%%%%%%%%%%%%%%%%%%%%%%%%%%%%%%%%
%
%\begin{frame}
%\frametitle{Hausaufgabe}
%
%\end{frame}
%
%
%%%%%%%%%%%%%%%%%%%%%%%%%%%%%%%%%%
%%%%%%%%%%%%%%%%%%%%%%%%%%%%%%%%%%
%\subsection*{Lösung der Hausaufgabe}
%
%%%%%%%%%%%%%%%%%%%%%%%%%%%%%%%%%%
%
%\begin{frame}
%\frametitle{Lösung}
%
%\end{frame}
%
%}%% END LOESUNG	
%%%%%%%%%%%%%%%%%%%%%%%%%%%%%%%%%%



%%%%%%%%%%%%%%%%%%%%%%%%%%%%%%%%%%%%%%%%%%%%%%%%%%%%
%%%             Preamble's End                   
%%%%%%%%%%%%%%%%%%%%%%%%%%%%%%%%%%%%%%%%%%%%%%%%%%%% 

\begin{document}
	
	
%%%% ue-loesung
%%%% true: Übung & Lösungen (slides) / false: nur Übung (handout)
%	\toggletrue{ue-loesung}

%%%% ha-loesung
%%%% true: Hausaufgabe & Lösungen (slides) / false: nur Hausaufgabe (handout)
%	\toggletrue{ha-loesung}

%%%% toc
%%%% true: TOC am Anfang von Slides / false: keine TOC am Anfang von Slides
\toggletrue{toc}

%%%% sectoc
%%%% true: TOC für Sections / false: keine TOC für Sections (StM handout)
%	\toggletrue{sectoc}

%%%% gliederung
%%%% true: Gliederung für Sections / false: keine Gliederung für Sections
%	\toggletrue{gliederung}


%%%%%%%%%%%%%%%%%%%%%%%%%%%%%%%%%%%%%%%%%%%%%%%%%%%%
%%%             Metadata                         
%%%%%%%%%%%%%%%%%%%%%%%%%%%%%%%%%%%%%%%%%%%%%%%%%%%%      

\title{Grundkurs Linguistik}

\subtitle{Lösungen -- Syntax IV}

\author[A. Machicao y Priemer]{
	{\small Antonio Machicao y Priemer}
	\\
	{\footnotesize \url{http://www.linguistik.hu-berlin.de/staff/amyp}}
	%	\\
	%	\href{mailto:mapriema@hu-berlin.de}{mapriema@hu-berlin.de}}
}

\institute{Institut für deutsche Sprache und Linguistik}


% bitte lassen, sonst kann man nicht sehen, von wann die PDF-Datei ist.
%\date{ }

%\publishers{\textbf{6. linguistischer Methodenworkshop \\ Humboldt-Universität zu Berlin}}

%\hyphenation{nobreak}


%%%%%%%%%%%%%%%%%%%%%%%%%%%%%%%%%%%%%%%%%%%%%%%%%%%%
%%%             Preamble's End                  
%%%%%%%%%%%%%%%%%%%%%%%%%%%%%%%%%%%%%%%%%%%%%%%%%%%%      


%%%%%%%%%%%%%%%%%%%%%%%%%      
\huberlintitlepage[22pt]
\iftoggle{toc}{
	\frame{
		\begin{multicols}{2}
			\frametitle{Inhaltsverzeichnis}
			\tableofcontents
			%[pausesections]
			\columnbreak
%			\textcolor{white}{
%				\ea \label{ex:05cHA2}
%					\ea \label{ex:05cHA2a}
%					\ex \label{ex:05cHA2b}
%					\ex \label{ex:05cHA2c}
%					\z
%				\ex \label{ex:05cHA3}
%					\ea \label{ex:05cHA3a}
%					\ex \label{ex:05cHA3b}
%					\ex \label{ex:05cHA3c}
%					\ex \label{ex:05cHA3d}
%					\ex \label{ex:05cHA3e}
%					\z
%				\ex \label{ex:05cHA5}
%					\ea \label{ex:05cHA5a}
%					\ex \label{ex:05cHA5b}
%					\z
%				\ex \label{ex:05cHA6}
%				\ex \label{ex:05cHA7}
%				\ex \label{ex:05cHA9}
%				\z
%			}
		\end{multicols}
	}
}


%%%%%%%%%%%%%%%%%%%%%%%%%%%%%%%%%%%
%%%%%%%%%%%%%%%%%%%%%%%%%%%%%%%%%%%
%\section{Übungen}


%%%%%%%%%%%%%%%%%%%%%%%%%%%%%%%%%%%
%%%%%%%%%%%%%%%%%%%%%%%%%%%%%%%%%%%
\section{Hausaufgaben}

%%%%%%%%%%%%%%%%%%
%06d Syntax ha-loesung
%%%%%%%%%%%%%%%%%%

\begin{frame}
\frametitle{Hausaufgabe -- Lösung} 

\begin{itemize}
	\item Bestimmen Sie den Kopf der folgenden \textbf{markierten} Phrasen und begründen Sie Ihre Entscheidung:
\end{itemize}
	
	\ea Es geht um [\textbf{wirklich von dieser Sache überzeugte und engagierte junge Schüler, die sich dennoch über das übliche und akzeptable Ausmaß hinaus daneben benommen haben}].
	\z

\pause 

	
\begin{itemize}
	\item \alertgreen{Kopf: Schüler (vorläufig)}
	\item \alertgreen{Interpretation: Es geht um \emph{Schüler}}
	\item \alertgreen{Distribution: Innerhalb einer PP wird eine (DP/)NP selegiert. Der Kopf dieser NP ist das Nomen.}
	\item \alertgreen{Phrasenaufbau: Alle anderen Modifikatoren beziehen sich auf das Nomen. }
\end{itemize}

	
\end{frame}


\begin{frame}
\frametitle{Hausaufgabe -- Lösung} 

\begin{itemize}
	\item Bestimmen Sie den Kopf der folgenden Phrase und begründen Sie Ihre Entscheidung:
\end{itemize}

\ea Wir warteten auf [\textbf{den von sich sehr überzeugten Redner}].
\z 

\pause 

\begin{itemize}
	\item \alertgreen{Kopf: Redner (vorläufig)}
	\item \alertgreen{Interpretation: Es geht um \emph{Redner}}
	\item \alertgreen{Distribution: Innerhalb einer PP wird eine (DP/)NP selegiert. Der Kopf dieser NP ist das Nomen.}
	\item \alertgreen{Phrasenaufbau: Alle anderen Modifikatoren beziehen sich auf das Nomen.}
\end{itemize}
	
	
\end{frame}


\begin{frame}
\frametitle{Hausaufgabe -- Lösung} 

\begin{itemize}
	
	\item Geben Sie für die folgenden Wörter den Subkategorisierungsrahmen (in dem besprochenen Format) und ein(en) Beispiel(satz), der den von Ihnen angegebenen Subkategorisierungsrahmen illustriert, an:
	
	\settowidth \jamwidth{[Lesart 1]}
	
	\ea übergeben: \pause \alertgreen{DP$_{\textsc{nom,quelle}}$ (DP$_{\textsc{dat,ziel}}$) DP$_{\textsc{akk,th}}$ $\underline{\qquad}$} \jambox{\alertgreen{[Lesart 1]}}
	
	\alertgreen{(dass) ich Peter die Briefe \emph{übergebe}}
	\pause 
	
	\ex stolz: \pause \alertgreen{PP$_{auf+\textsc{akk,th}}$ $\underline{\qquad}$ (DP$_{\textsc{exp}}$)}
	
	\alertgreen{die auf ihre Tochter \emph{stolze} Mutter}
	\pause 
	
	\ex donnern: \pause \alertgreen{DP$_{(es),\textsc{nom}}$ $\underline{\qquad}$}
	
	\alertgreen{(dass) es \emph{donnert}}
	
	\ex glauben: \pause \alertgreen{DP$_{\textsc{nom, exp}}$ $\underline{\qquad}$ PP$_{an+\textsc{thema}}$}
	
	\alertgreen{Peter \emph{glaubt} an den Weltfrieden}
	
	\z 
	
	%	wissen, regnen, scheinen, mit, ängstigen, fürchten, Banane, Tatsache, warten, Angst, wohnen, bewohnen, auf, malen, krank, bemalen, fragen
	
\end{itemize}

\end{frame}


\begin{frame}
\frametitle{Hausaufgabe -- Lösung} 

\begin{itemize}

\item Geben Sie für die folgenden Wörter den Subkategorisierungsrahmen (in dem besprochenen Format) und ein(en) Beispiel(satz), der den von Ihnen angegebenen Subkategorisierungsrahmen illustriert, an:

\settowidth \jamwidth{[Lesart 1]}

\ea Frage: \pause \alertgreen{DP$_{\textsc{gen,ag}}$ $\underline{\qquad}$ PP$_{nach+\textsc{dat,th}}$}

\alertgreen{die Frage nach dem Schatz}
\pause 

\ex erschrecken: \pause \alertgreen{DP$_{\textsc{nom,causer}}$ DP$_{\textsc{akk,exp}}$ $\underline{\qquad}$} \jambox{\alertgreen{[Lesart 1]}}

\alertgreen{(dass) Maria Peter erschreckt}
\pause 

\ex bemalen: \pause \alertgreen{DP$_{\textsc{nom,ag}}$ DP$_{\textsc{akk,pat/th}}$ $\underline{\qquad}$}

\alertgreen{(dass) ich die Wand bemale}
\z 

%	wissen, regnen, scheinen, mit, ängstigen, fürchten, Banane, Tatsache, warten, Angst, wohnen, bewohnen, auf, malen, krank, bemalen, fragen

\end{itemize}

\end{frame}

\begin{frame}
\frametitle{Hausaufgabe -- Lösung} 

\begin{itemize}
	
	\item Bestimmen Sie in den folgenden Sätzen, welche Phrasen Argumente und welche Modifikatoren des Verbs sind, und begründen Sie Ihre Entscheidung.
	
\end{itemize}

\ea Maria bearbeitete die Folien mit sehr viel Kreativität.

\pause 
\alertgreen{Arg.: Maria, die Folien}

\alertgreen{Mod.: mit sehr viel Kreatitvität}

\alertgreen{Begründung: \dots }

\pause 


\ex Maria arbeitete an den Folien den ganzen Tag.

\pause 
\alertgreen{Arg.: Maria, an den Folien}

\alertgreen{Mod.: den ganzen Tag}

\alertgreen{Begründung: \dots }

\z 

\end{frame}


\begin{frame}
\frametitle{Hausaufgabe -- Lösung} 

\begin{itemize}

\item Bestimmen Sie in den folgenden Sätzen, welche Phrasen Argumente und welche Modifikatoren des Verbs sind, und begründen Sie Ihre Entscheidung.

\end{itemize}

\settowidth \jamwidth{[Lesart 2]}

\ea Peter wirkte auf seinen Sohn stolz. \jambox{\alertgreen{[Lesart 1]}}

\pause 
\alertgreen{Arg.: Peter, auf seinen Sohn stolz}

\alertgreen{Mod.: $\emptyset$}

\alertgreen{Begründung: \dots }

\pause 


\ex Peter wirkte auf seinen Sohn stolz. \jambox{\alertgreen{[Lesart 2]}}

\pause 
\alertgreen{Arg.: Peter, auf seinen Sohn, stolz}

\alertgreen{Mod.: $\emptyset$}

\alertgreen{Begründung: \dots }
\z 


\end{frame}

%% -*- coding:utf-8 -*-

%%%%%%%%%%%%%%%%%%%%%%%%%%%%%%%%%%%%%%%%%%%%%%%%%%%%%%%%%


\def\insertsectionhead{\refname}
\def\insertsubsectionhead{}

\huberlinjustbarfootline


\ifpdf
\else
\ifxetex
\else
\let\url=\burl
\fi
\fi
\begin{multicols}{2}
{\tiny
%\beamertemplatearticlebibitems

\bibliography{../gkbib,../bib-abbr,../biblio}
\bibliographystyle{../unified}
}
\end{multicols}





%% \section{Literatur}
%% \begin{frame}[allowframebreaks]
%% \frametitle{Literatur}
%% 	\footnotesize

%% \bibliographystyle{unified}

%% 	%German
%% %	\bibliographystyle{deChicagoMyP}

%% %	%English
%% %	\bibliographystyle{chicago} 

%% 	\bibliography{gkbib,bib-abbr,biblio}
	
%% \end{frame}



\end{document}
