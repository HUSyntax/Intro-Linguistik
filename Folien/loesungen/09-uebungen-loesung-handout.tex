%%%%%%%%%%%%%%%%%%%%%%%%%%%%%%%%%%%%%%%%%%%%%%
%% Compile: XeLaTeX BibTeX XeLaTeX XeLaTeX
%% Loesung-Handout: Antonio Machicao y Priemer
%% Course: GK Linguistik
%%%%%%%%%%%%%%%%%%%%%%%%%%%%%%%%%%%%%%%%%%%%%%

%\documentclass[a4paper,10pt, bibtotoc]{beamer}
\documentclass[10pt,handout]{beamer}

%%%%%%%%%%%%%%%%%%%%%%%%
%%     PACKAGES      
%%%%%%%%%%%%%%%%%%%%%%%%

%%%%%%%%%%%%%%%%%%%%%%%%
%%     PACKAGES       %%
%%%%%%%%%%%%%%%%%%%%%%%%



%\usepackage[utf8]{inputenc}
%\usepackage[vietnamese, english,ngerman]{babel}   % seems incompatible with german.sty
%\usepackage[T3,T1]{fontenc} breaks xelatex

\usepackage{lmodern}
\usepackage{calligra}

\usepackage{amsmath}
\usepackage{amsfonts}
\usepackage{amssymb}
%% MnSymbol: Mathematische Klammern und Symbole (Inkompatibel mit ams-Packages!)
%% Bedeutungs- und Graphemklammern: $\lsem$ Tisch $\rsem$ $\langle TEXT \rangle$ $\llangle$ TEXT $\rrangle$ 
\usepackage{MnSymbol}
%% ulem: Strike out
\usepackage[normalem]{ulem}  

%% Special Spaces (s. Commands)
\usepackage{xspace}				
\usepackage{setspace}
%	\onehalfspacing

%% mdwlist: Special lists
\usepackage{mdwlist}	


%%%%%%%%%%%%%%%%%%%%%%%%%%%%%%%%
%% TIPA & Phonetics

\usepackage[
%noenc,
safe]{tipa}

%% TIPA Problems/Solutions:
%% Problems with U, serif fonts and ligatures

%%Test 1
%\DeclareFontSubstitution{T3}{cmss}{m}{n}

%%Test 2
%\DeclareFontSubstitution{T3}{ptm}{m}{n}

%%Test 3
%\usepackage{tipx}


%\usepackage{vowel}


%%%%%%%%%%%%%%%%%%%%%%%%%%%%%%%%
%% Examples

\usepackage{jambox}



%\usepackage{forest-v105}
%\usepackage{langsci-forest-v105-setup}


%%%%%%%%%%%%%%%%%%%%%%%%%%%%%%%%
%% Fonts for Chinese, Vietnamese, etc. (s. Graphematik)

\usepackage{xeCJK}
\setCJKmainfont{SimSun}


%\usepackage{natbib}
%\setcitestyle{notesep={:~}}


% for toggles, is loaded in hu-beamer-includes-pdflatex
%\usepackage{etex}


%%%%%%%%%%%%%%%%%%%%%%%%%%%%%%%%
%% Fonts for Fraktur

\usepackage{yfonts}

\usepackage{url}

% für UDOP
\usepackage{adjustbox}


%% huberlin: Style sheet
%\usepackage{huberlin}
\usepackage{hu-beamer-includes-pdflatex}
\huberlinlogon{0.86cm}

% %% % use this definition, if you want to see the outlines in the handout
\renewcommand{\outline}[1]{%
%\beamertemplateemptyfootbar%
\huberlinjustbarfootline
\frame{\frametitle{\outlineheading}#1}%
%\beamertemplatecopyrightfootframenumber%
\huberlinnormalfootline 
\huberlinpagedec
}



%% Last Packages
%\usepackage{hyperref}	%URLs
%\usepackage{gb4e}		%Linguistic examples

% sorry this was incompatible with gb4e and had to go.
%\usepackage{linguex-cgloss}	%Linguistic examples (patched version that works with jambox

\usepackage{multirow}  %Mehrere Zeilen in einer Tabelle
\usepackage{adjustbox} %adjusting tables
%\usepackage{array}
\usepackage{marginnote}	%Notizen




%%%%%%%%%%%%%%%%%%%%%%%%%%%%%%%%%%%%%%%%%%%%%%%%%%%%
%%%            MyP-Commands                     
%%%%%%%%%%%%%%%%%%%%%%%%%%%%%%%%%%%%%%%%%%%%%%%%%%%%


%%%%%%%%%%%%%%%%%%%%%%%%%%%%%%%%
% Delete Caption from Figures and Tables
\setbeamertemplate{caption}{\centering\insertcaption\par }


%%%%%%%%%%%%%%%%%%%%%%%%%%%%%%%%
% German quotation marks:
\newcommand{\gqq}[1]{\glqq{}#1\grqq{}}		%double
\newcommand{\gq}[1]{\glq{}#1\grq{}}			%simple


%%%%%%%%%%%%%%%%%%%%%%%%%%%%%%%%
% Abbreviations in German
% package needed: xspace
% Short space in German abbreviations: \,	
\newcommand{\idR}{\mbox{i.\,d.\,R.}\xspace}
\newcommand{\su}{\mbox{s.\,u.}\xspace}
%\newcommand{\ua}{\mbox{u.\,a.}\xspace}       % in abbrev
\newcommand{\vgl}{\mbox{vgl.}\xspace}       % in abbrev
%\newcommand{\zB}{\mbox{z.\,B.}\xspace}       % in abbrev
%\newcommand{\s}{s.~}
%not possibel: \dh --> d.\,h.

%rot unterstrichen
%\newcommand{\rotul}[1]{\textcolor{red}{\underline{#1}}}

%%%%%%%%%%%%%%%%%%%%%%%%%%%%%%%%
%Abbreviations in English
\newcommand{\ao}{a.o.\ }	% among others
%\newcommand{\cf}[1]{(cf.~#1)}	% confer = compare
\renewcommand{\ia}{i.a.}	% inter alia = among others
%\newcommand{\ie}{i.e.~}	% id est = that is
\newcommand{\fe}{e.g.~}	% exempli gratia = for example
%not possible: \eg --> e.g.~
\newcommand{\vs}{vs.\ }	% versus
\newcommand{\wrt}{w.r.t.\ }	% with respect to


%%%%%%%%%%%%%%%%%%%%%%%%%%%%%%%%
% Dash:
\newcommand{\gs}[1]{--\,#1\,--}


%%%%%%%%%%%%%%%%%%%%%%%%%%%%%%%%
% Rightarrow with and without space
\def\ra{\ensuremath\rightarrow}			%without space
\def\ras{\ensuremath\rightarrow\ }		%with space


%%%%%%%%%%%%%%%%%%%%%%%%%%%%%%%%
%% X-bar notation

%% Notation with primes (not emphasized): \xbar{X}
\newcommand{\MyPxbar}[1]{#1$^{\prime}$}
\newcommand{\xxbar}[1]{#1$^{\prime\prime}$}
\newcommand{\xxxbar}[1]{#1$^{\prime\prime\prime}$}

%% Notation with primes (emphasized): \exbar{X}
\newcommand{\exbar}[1]{\emph{#1}$^{\prime}$}
\newcommand{\exxbar}[1]{\emph{#1}$^{\prime\prime}$}
\newcommand{\exxxbar}[1]{\emph{#1}$^{\prime\prime\prime}$}

% Notation with zero and max (not emphasized): \xbar{X}
\newcommand{\zerobar}[1]{#1$^{0}$}
\newcommand{\maxbar}[1]{#1$^{\textsc{max}}$}

% Notation with zero and max (emphasized): \xbar{X}
\newcommand{\ezerobar}[1]{\emph{#1}$^{0}$}
\newcommand{\emaxbar}[1]{\emph{#1}$^{\textsc{max}}$}

%% Notation with bars (already implemented in gb4e):
% \obar{X}, \ibar{X}, \iibar{X}, \mbar{X} %Problems with \mbar!
%
%% Without gb4e:
\newcommand{\overbar}[1]{\mkern 1.5mu\overline{\mkern-1.5mu#1\mkern-1.5mu}\mkern 1.5mu}
%
%% OR:
\newcommand{\MyPibar}[1]{$\overline{\textrm{#1}}$}
\newcommand{\MyPiibar}[1]{$\overline{\overline{\textrm{#1}}}$}
%% (emphasized):
\newcommand{\eibar}[1]{$\overline{#1}$}
\newcommand{\eiibar}[1]{\overline{$\overline{#1}}$}

%%%%%%%%%%%%%%%%%%%%%%%%%%%%%%%%
%% Subscript & Superscript: no italics
\newcommand{\MyPdown}[1]{\textsubscript{#1}}
\newcommand{\MyPup}[1]{\textsuperscript{#1}}

%%%%%%%%%%%%%%%%%%%%%%%%%%%%%%%%
%% Small caps subscripts & superscripts
\newcommand{\scdown}[1]{\textsubscript{\textsc{#1}}}
\newcommand{\scup}[1]{\textsuperscript{\textsc{#1}}}

%%%%%%%%%%%%%%%%%%%%%%%%%%%%%%%%
% Objekt language marking:
%\newcommand{\obj}[1]{\glqq{}#1\grqq{}}	%German double quotes
%\newcommand{\obj}[1]{``#1''}			%English double quotes
%\newcommand{\MyPobj}[1]{\emph{#1}}		%Emphasising
\newcommand{\MyPobj}[1]{\textit{#1}}		%Emphasising

%%%%%%%%%%%%%%%%%%%%%%%%%%%%%%%%
% Size:
\newcommand{\size}[1]{#1}	% f.e. resize citations


%%%%%%%%%%%%%%%%%%%%%%%%%%%%%%%%
%% Semantic types (<e,t>), features, variables and graphemes in angled brackets 

%%% types and variables, in math mode: angled brackets + italics + no space
%\newcommand{\type}[1]{$<#1>$}

%%% OR more correctly: 
%%% types and variables, in math mode: chevrons! + italics + no space
\newcommand{\MyPtype}[1]{$\langle #1 \rangle$}

%%% features and graphemes, in math mode: chevrons! + italics + no space
\newcommand{\abe}[1]{$\langle #1 \rangle$}


%%% features and graphemes, in math mode: chevrons! + no italics + space
\newcommand{\ab}[1]{$\langle$#1$\rangle$}  %%same as \abu  
\newcommand{\abu}[1]{$\langle$#1$\rangle$} %%Umlaute


%% Presuppositions
\newcommand{\prspp}{$\gg$} 

%% Implicature
\newcommand{\implc}{$+ \mkern-5mu >$} 

%% Enttailment
\newcommand{\ent}{$\vDash$}

%% Other semantic symbols: 
%% entailment: $\Rightarrow$ $\vDash$
%% equivalence: $\Leftrightarrow$ $\equiv$
%% biconditional: $\leftrightarrow$ 
%% lexical rule: $\mapsto$
%% greater/less/equal: $>$ $\geq$ $<$ $\leq$
%% definition: $:=$ $=$\textsubscript{def}


%%%%%%%%%%%%%%%%%%%%%%%%%%%%%%%%
% Marking text with colour:
% package needed: xcolor
% Command \alert{} in Beamer >> FU-grün (leider!! @Stefan)

%%%neue Farbbefehle in Anlehnung an rotul
%%%(s. hu-beamer-includes-pdflatex.sty in texmf)

%% Farbdefinitionen:

\definecolor{HUred}{RGB}{138,15,20}
\definecolor{HUblue}{RGB}{0,55,108}
\definecolor{HUgreen}{RGB}{0,87,44}

%\newcommand{\alertred}[1]{\textcolor{red}{#1}}  % basic red
\newcommand<>{\alertred}[1]{{\color#2[RGB]{138,15,20}#1}}  %HU rot + overlay

%\newcommand{\alertblue}[1]{\textcolor{blue}{#1}} 		% basic blue
\newcommand<>{\alertblue}[1]{{\color#2[RGB]{0,55,108}#1}} %HU blue + overlay

%\newcommand{\alertgreen}[1]{\textcolor{green}{#1}}	% basic green
\newcommand<>{\alertgreen}[1]{{\color#2[RGB]{0,87,44}#1}} %HU green + overlay


%%% Verwendung der oben definierten Farben mit Unterschied in Handout und Beamer:

\mode<handout>{%
	\newcommand<>{\hured}[1]{\only#2{\underline{#1}}}
	\newcommand<>{\hublue}[1]{\only#2{\textbf{#1}}}
	\newcommand<>{\hugreen}[1]{\only#2{\textsc{#1}}}
}
%
\mode<beamer>{%
	\newcommand<>{\hured}[1]{\alertred#2{#1}}
	\newcommand<>{\hublue}[1]{\alertblue#2{#1}}
	\newcommand<>{\hugreen}[1]{\alertgreen#2{#1}}
}


%%%%%%%%%%%%%%%%%%%%%%%%%%%%%%%%
%% Outputbox
\newcommand{\outputbox}[1]{\noindent\fbox{\parbox[t][][t]{0.98\linewidth}{#1}}\vspace{0.5em}}


%%%%%%%%%%%%%%%%%%%%%%%%%%%%%%%%
%% (Syntactic) Trees
% package needed: forest
%
%% Setting for simple trees
\forestset{
	MyP edges/.style={for tree={parent anchor=south, child anchor=north}}
}

%% this is taken from langsci-setup file
%% Setting for complex trees
%% \forestset{
%% 	sm edges/.style={for tree={parent anchor=south, child anchor=north,align=center}}, 
%% background tree/.style={for tree={text opacity=0.2,draw opacity=0.2,edge={draw opacity=0.2}}}
%% }

\newcommand\HideWd[1]{%
	\makebox[0pt]{#1}%
}

%%%%%%%%%%%%%%%%%%%%%%%%%%%%%%%
%%solutions in green + w/ jambox
\newcommand{\loesung}[2]{\jambox{\visible<#1->{\alertgreen{#2}}}}

%%%%%%%%%%%%%%%%%%%%%%%%%%%%%%%%
%% TIPA Lösungen           

%%Tipa serif font fixed (requires package 'Linux Libertine B')

%% Solution 1 (RF)
%% Tipa font:
%\renewcommand\textipa[1]{{\fontfamily{cmr}\tipaencoding #1}}

%% Solution 2 (RF): older code for texlive 2017?
%\newfontfamily{\tipacm}[Scale=MatchUppercase]{Linux Libertine B}
%\renewcommand\useTIPAfont{\tipacm}

%\NewEnviron{IPA}{\expandafter\textipa\expandafter{\BODY}} %% not needed anymore

%% Solution 3 (RF): this solution is working but with problems with ligatures
%%% works for texlive 2018
\newfontfamily{\ipafont}[Scale=MatchUppercase]{Linux Libertine B}
\def\useTIPAfont{\ipafont}

%% Solution 4 (Kopecky & MyP): Test package: tipx (s. localpackages) and comment "Solution 3" 


%%%%%%%%%%%%%%%%%%%%%%%%%%%%%%%%
%% Toggles                  


\newtoggle{uebung}
\newtoggle{loesung}\togglefalse{loesung}

\newtoggle{hausaufgabe}

%\newtoggle{ha-loesung}\togglefalse{ha-loesung}
\newtoggle{phonologie-loesung}
\newtoggle{graphematik-loesung}


%% Neue Toggle-Struktur
\newtoggle{toc}
\newtoggle{sectoc}
\newtoggle{gliederung}

\newtoggle{ue-loesung}
\newtoggle{ha-loesung}
%%

% The toc is not needed on Handouts. Save trees.
\mode<handout>{
\togglefalse{toc}
}

\newtoggle{hpsgvorlesung}\togglefalse{hpsgvorlesung}
\newtoggle{syntaxvorlesungen}\togglefalse{syntaxvorlesungen}

%\includecomment{psgbegriffe}
%\excludecomment{konstituentenprobleme}
%\includecomment{konstituentenprobleme-hinweis}

\newtoggle{konstituentenprobleme}\togglefalse{konstituentenprobleme}
\newtoggle{konstituentenprobleme-hinweis}\toggletrue{konstituentenprobleme-hinweis}

%\includecomment{einfsprachwiss-include}
%\excludecomment{einfsprachwiss-exclude}
\newtoggle{einfsprachwiss-include}\toggletrue{einfsprachwiss-include}
\newtoggle{einfsprachwiss-exclude}\togglefalse{einfsprachwiss-exclude}

\newtoggle{psgbegriffe}\toggletrue{psgbegriffe}

\newtoggle{gb-intro}\togglefalse{gb-intro}


%%%%%%%%%%%%%%%%%%%%%%%%%%%%%%%%
%% Useful commands                    

%%%%%%%%%%%%%%%%%%%%%
%% FOR ITEMS:
%\begin{itemize}
%  \item<2-> from point 2
%  \item<3-> from point 3 
%  \item<4-> from point 4 
%\end{itemize}
%
% or: \onslide<2->
% or \only<2->{Text}
% or: \pause

%%%%%%%%%%%%%%%%%%%%%
%% VERTICAL SPACE:
% \vspace{.5cm}
% \vfill

%%%%%%%%%%%%%%%%%%%%%
% RED MARKING OF TEXT:
%\alert{bis spätestens Mittwoch, 18 Uhr}
%\newcommand{\alertred}[1]{\textcolor{red}{#1}}

%%%%%%%%%%%%%%%%%%%%%
%% RESCALE BIG TABLES:
%\scalebox{0.8}{
%For Big Tables
%}

%%%%%%%%%%%%%%%%%%%%%
%% BLOCKS:
%\begin{alertblock}{Title}
%Text
%\end{alertblock}
%
%\begin{block}{Title}
%Text
%\end{block}
%
%\begin{exampleblock}{Title}
%Text
%\end{exampleblock}

%%%%%%%%%%%%%%%%%%%%%
%% JAMBOX FOR EXAMPLES:
%\ea 
%\settowidth\jamwidth{Test} 
%Die Studierenden, die weitgehend von Stipendien leben, erhalten einen Mietzuschuss. 
%\jambox{Test}
%\z 

%%%%%%%%%%%%%%%%%%%%%
%% TOGGLES:


%%%%%%%%%%%%%%%%%%%%%%%%%%%%%%%%%%
%%%%%%%%%%%%%%%%%%%%%%%%%%%%%%%%%%
%\subsection{Übung}
%
%%%%%%%%%%%%%%%%%%%%%%%%%%%%%%%%%%
%%%%%%%%%%%%%%%%%%%%%%%%%%%%%%%%%%
%\iftoggle{uebung}{
%%%%%%%%%%%%%%%%%%%%%%%%%%%%%%%%%%
%\begin{frame}
%\frametitle{Übung}
%
%\end{frame}
%
%} 
%%% END true = Q
%%% BEGIN false = Q + A
%{
%%%%%%%%%%%%%%%%%%%%%%%%%%%%%%%%%%
%\begin{frame}
%\frametitle{Übung}
%
%\end{frame}
%%%%%%%%%%%%%%%%%%%%%%%%%%%%%%%%%%
%
%\begin{frame}
%\frametitle{Lösung}
%
%\end{frame}
%
%}%% END LOESUNG	
%%%%%%%%%%%%%%%%%%%%%%%%%%%%%%%%%%


%%%%%%%%%%%%%%%%%%%%%%%%%%%%%%%%%%
%%%%%%%%%%%%%%%%%%%%%%%%%%%%%%%%%%
%\subsection{Hausaufgabe}
%
%%%%%%%%%%%%%%%%%%%%%%%%%%%%%%%%%%
%%%%%%%%%%%%%%%%%%%%%%%%%%%%%%%%%%
%\iftoggle{hausaufgabe}{
%%%%%%%%%%%%%%%%%%%%%%%%%%%%%%%%%%
%
%\begin{frame}
%\frametitle{Hausaufgabe}
%
%\end{frame}
%
%} 
%%% END true = Q
%%% BEGIN false = Q + A
%{
%%%%%%%%%%%%%%%%%%%%%%%%%%%%%%%%%%
%
%\begin{frame}
%\frametitle{Hausaufgabe}
%
%\end{frame}
%
%
%%%%%%%%%%%%%%%%%%%%%%%%%%%%%%%%%%
%%%%%%%%%%%%%%%%%%%%%%%%%%%%%%%%%%
%\subsection*{Lösung der Hausaufgabe}
%
%%%%%%%%%%%%%%%%%%%%%%%%%%%%%%%%%%
%
%\begin{frame}
%\frametitle{Lösung}
%
%\end{frame}
%
%}%% END LOESUNG	
%%%%%%%%%%%%%%%%%%%%%%%%%%%%%%%%%%



%%%%%%%%%%%%%%%%%%%%%%%%%%%%%%%%%%%%%%%%%%%%%%%%%%%%
%%%             Preamble's End                   
%%%%%%%%%%%%%%%%%%%%%%%%%%%%%%%%%%%%%%%%%%%%%%%%%%%% 

\begin{document}
	
	
%%%% ue-loesung
%%%% true: Übung & Lösungen (slides) / false: nur Übung (handout)
%	\toggletrue{ue-loesung}

%%%% ha-loesung
%%%% true: Hausaufgabe & Lösungen (slides) / false: nur Hausaufgabe (handout)
%	\toggletrue{ha-loesung}

%%%% toc
%%%% true: TOC am Anfang von Slides / false: keine TOC am Anfang von Slides
\toggletrue{toc}

%%%% sectoc
%%%% true: TOC für Sections / false: keine TOC für Sections (StM handout)
%	\toggletrue{sectoc}

%%%% gliederung
%%%% true: Gliederung für Sections / false: keine Gliederung für Sections
%	\toggletrue{gliederung}


%%%%%%%%%%%%%%%%%%%%%%%%%%%%%%%%%%%%%%%%%%%%%%%%%%%%
%%%             Metadata                         
%%%%%%%%%%%%%%%%%%%%%%%%%%%%%%%%%%%%%%%%%%%%%%%%%%%%      

\title{Grundkurs Linguistik}

\subtitle{Lösungen -- Übungen}

\author[A. Machicao y Priemer]{
	{\small Antonio Machicao y Priemer}
	\\
	{\footnotesize \url{http://www.linguistik.hu-berlin.de/staff/amyp}}
	%	\\
	%	\href{mailto:mapriema@hu-berlin.de}{mapriema@hu-berlin.de}}
}

\institute{Institut für deutsche Sprache und Linguistik}


% bitte lassen, sonst kann man nicht sehen, von wann die PDF-Datei ist.
%\date{ }

%\publishers{\textbf{6. linguistischer Methodenworkshop \\ Humboldt-Universität zu Berlin}}

%\hyphenation{nobreak}


%%%%%%%%%%%%%%%%%%%%%%%%%%%%%%%%%%%%%%%%%%%%%%%%%%%%
%%%             Preamble's End                  
%%%%%%%%%%%%%%%%%%%%%%%%%%%%%%%%%%%%%%%%%%%%%%%%%%%%      


%%%%%%%%%%%%%%%%%%%%%%%%%      
\huberlintitlepage[22pt]
\iftoggle{toc}{
	\frame{
\frametitle{Inhaltsverzeichnis}

		\begin{multicols}{6}
			\tableofcontents
			%[pausesections]
			\columnbreak
			\textcolor{white}{
				\ea\label{ex:01}
				\ex\label{ex:02}
				\ex\label{ex:03}
				\ex\label{ex:04}
				\ex\label{ex:05}
				\ex\label{ex:06}
				\ex\label{ex:07}
				\ex\label{ex:08}
				\ex\label{ex:09}
				\ex\label{ex:10}
				\ex\label{ex:11}
				\ex\label{ex:12}
				\ex\label{ex:13}
				\ex\label{ex:14}
				\ex\label{ex:15}
				\ex\label{ex:16}
				\ex\label{ex:17}
				\ex\label{ex:18}
				\ex\label{ex:19}
				\ex\label{ex:20}
				\ex\label{ex:21}
				\ex\label{ex:22}
				\ex\label{ex:23}
				\ex\label{ex:24}
				\ex\label{ex:25}
				\ex\label{ex:26}
				\ex\label{ex:27}
				\ex\label{ex:28}
				\ex\label{ex:29}
				\z
			}
		\end{multicols}
	}
}


%%%%%%%%%%%%%%%%%%%%%%%%%%%%%%%%%%%
%%%%%%%%%%%%%%%%%%%%%%%%%%%%%%%%%%%
\section{Phonetik/Phonologie}

%%%%%%%%%%%%%%%%%%%%%%%%%%%%%%%%%%
%% UE 1 - 09 Übungen
%%%%%%%%%%%%%%%%%%%%%%%%%%%%%%%%%%

\begin{frame}
\frametitle{Übung: Phonetik/Phonologie -- Lösung}

\begin{itemize}
	\item[1.] Erläutern Sie den Unterschied zwischen Phon, Phonem und Allophon.
	
	\item Phon:
		
		\only<2->{
		\begin{itemize}
			\item \alertgreen{Minimaleinheit der Phonetik}
			\item \alertgreen{physikalisch messbare lautliche Einheit einer Sprache} \pause
		\end{itemize}
	}
		
	\item Phonem:
		
		\only<3->{
		\begin{itemize}
			\item \alertgreen{Minimaleinheit der Phonologie}
			\item \alertgreen{abstraktes Konstrukt, steht für eine Menge von möglichen Phonen}
			\item \alertgreen{ermittelbar durch Minimalpaarbildung (strukturalistisches Kriterium)} \pause
		\end{itemize}
	}
		
	\item Allophon:
		
		\only<4->{
		\begin{itemize}
			\item \alertgreen{phonetische Realisierungsvariante eines Phonems}
			\item \alertgreen{Untertypen: komplementäre und freie Allophonie, regionale und soziale Variation}
		\end{itemize}
		}

\end{itemize}

\end{frame}

%%%%%%%%%%%%%%%%%%%%%%%%%%%%%%%%%%%

\begin{frame}
	
\begin{itemize}
	\item[2.] Geben sie die artikulatorischen Eigenschaften der folgenden Laute an.
	
	\begin{exe}
		\exr{ex:01}
	\settowidth \jamwidth{\alertgreen{halbhoher fast vorderer ungerundeter ungespannter Vokal}}
	
		\begin{xlist}
			\ex \textipa{[r]}	\only<2->{\jambox{\alertgreen{alveolarer stimmhafter Vibrant}}}
			\ex \textipa{[P]}	\only<3->{\jambox{\alertgreen{glottaler stimmloser Plosiv}}}
			\ex \textipa{[b]}	\only<4->{\jambox{\alertgreen{bilabialer stimmhafter Plosiv}}}
			\ex \textipa{[f]}	 \only<5->{\jambox{\alertgreen{labiodentaler stimmloser Frikativ}}}
			\ex \textipa{[I]}	 \only<6->{\jambox{\alertgreen{halbhoher fast vorderer ungerundeter ungespannter Vokal}}}
			\ex \textipa{[u:]}	\only<7->{\jambox{\alertgreen{hoher hinterer gerundeter gespannter (langer) Vokal}}}
		\end{xlist}
	
	\end{exe}

\end{itemize}

\end{frame}

%%%%%%%%%%%%%%%%%%%%%%%%%%%%%%%%%%%

\begin{frame}
	
\begin{itemize}
	\item[3.] Geben Sie die phonologische Repräsentation und die phonetische standarddeutsche Transkription der folgenden Wörter mit Silbenstruktur und X-Skelettschicht an.
		
	\begin{exe}
		\exr{ex:02}
		
		\begin{xlist}
			\ex Näherinnen
			\ex Zwischendinger
			\ex königlich
		\end{xlist}
	
	\end{exe}

\end{itemize}

\end{frame}

%%%%%%%%%%%%%%%%%%%%%%%%%%%%%%%%%%%	
	
\begin{frame}

(2a): Näherinnen \\

\medskip

	\alertgreen{\textipa{/ne:.@.KI\d{n}@n/} \ras \textipa{[\textprimstress ne:.@.KI\d{n}@n]} }
	
	\centering
	\alertgreen{
	\scalebox{1}{
	\begin{forest} MyP edges, [,phantom
		[$\sigma$
		[O
			[x, tier=word[\textipa{n}]]
		]
		[R
			[N
				[x, tier=word[\textipa{E:}]]
			]
			[K]
		]
		]
		[$\sigma$
		[O]
		[R
			[N
				[x, tier=word[\textipa{@}]]
			]
			[K]
		]
		]
		[$\sigma$
		[O
			[x, tier=word[\textipa{K}]]+
		]
		[R
			[N
				[x, tier=word[\textipa{I}]]
			]
			[K
				[x, tier=word, name=x[\textipa{n}]]
			]
		]
		]
		[$\sigma$
		[O, name=O]
		[R
			[N
				[x, tier=word[\textipa{@}]]
			]
			[K
				[x, tier=word[\textipa{n}]]
			]
		]
		]
	]
	\draw[HUgreen] (O.south)--(x.north);
	\end{forest}
	}}

\end{frame}

%%%%%%%%%%%%%%%%%%%%%%%%%%%%%%%%%%%

\begin{frame}

(2b): Zwischendinger	\\

\medskip

	\alertgreen{\textipa{/\t{ts}vI\d{S}@n.dIn.g@\textscr/} \ras \textipa{[\textprimstress \t{ts}vI\d{S}@n.dI\d{N}5]}}
	
	\centering
	\alertgreen{
	\scalebox{1}{
	\begin{forest} MyP edges, [,phantom
		[$\sigma$
		[O
			[x, tier=word[\textipa{\t{ts}}] ]
			[x, tier=word[\textipa{v}] ]
		]
		[R
			[N
				[x, tier=word[\textipa{I}]]
			]
			[K
				[x, tier=word, name=x[\textipa{S}]]
			]
		]
		]
		[$\sigma$
		[O, name=O]
		[R
			[N
				[x, tier=word[\textipa{@}] ]
			]
			[K
				[x, tier=word[\textipa{n}]]
			]
		]
		]
		[$\sigma$
		[O
			[x, tier=word[\textipa{d}]]
		]
		[R
			[N
				[x, tier=word[\textipa{I}]]
			]
			[K
				[x, tier=word, name=x2[\textipa{N}]]
			]
		]
		]
		[$\sigma$
		[O, name=O2]
		[R
			[N
				[x, tier=word[\textipa{5}]]
			]
			[K]	
		]
		]
	]
	\draw[HUgreen] (O.south)--(x.north);
	\draw[HUgreen] (O2.south)--(x2.north);
	\end{forest}
	} }

\end{frame}

%%%%%%%%%%%%%%%%%%%%%%%%%%%%%%%%%%%

\begin{frame}

(2c): königlich \\

\medskip

	\alertgreen{\textipa{/k\o:.nIg.lI\c{c}/} \ras \textipa{[\textprimstress k\o:.nIk.lI\c{c}]}} ~\\
	
	\centering
	\alertgreen{
	\scalebox{1}{
	\begin{forest} MyP edges, [,phantom
		[$\sigma$
		[O
			[x, tier=word[\textipa{k}]]
		]
		[R
			[N
				[x, tier=word[\textipa{\o:}, name=o]]
				[x, tier=word, name=x]		
			]
			[K]
		]
		]
		[$\sigma$
		[O
			[x, tier=word[\textipa{n}]]
		]
		[R
			[N
				[x, tier=word[\textipa{I}]]
			]
			[K
				[x, tier=word[\textipa{k}]]
			]
		]
		]
		[$\sigma$
		[O
			[x, tier=word[\textipa{l}]]
		]
		[R
			[N
				[x, tier=word[\textipa{I}]]
			]
			[K
				[x, tier=word[\textipa{\c{c}}]]
			]
		]
		]
	]
	\draw[HUgreen] (x.south)--(o.north);		
	\end{forest}
	} }

\end{frame}

%%%%%%%%%%%%%%%%%%%%%%%%%%%%%%%%%%%
\begin{frame}

\begin{itemize}
	\item[4.] Benennen Sie die phonetisch/phonologischen Prozesse, die stattfinden, bei der Aussprache der folgenden Wörter:
	
	\begin{exe}
		\exr{ex:03}
	\settowidth \jamwidth{\alertgreen{regressive velare Nasalassimilation: \textipa{/n/} \ras \textipa{[N]}},}
	
		\begin{xlist}
			\ex mild	\only<2->{\jambox{\alertgreen{Auslautverhärtung: \textipa{/d/} \ras \textipa{[t]}}}}
\medskip
			\ex ungelenkig	\only<3->{\jambox{\alertgreen{Knacklauteinsetzung: $\emptyset$ \ras \textipa{[P]}},}}
			\only<3->{\jambox{\alertgreen{regressive velare Nasalassimilation: \textipa{/n/} \ras \textipa{[N]}},}} 
			\only<3->{\jambox{\alertgreen{g-Spirantisierung: \textipa{/g/} \ras \textipa{[\c{c}]}}}}
\medskip
			\ex süchtig	\only<4->{\jambox{\alertgreen{g-Spirantisierung: \textipa{/g/} \ras \textipa{[\c{c}]}}}}
\medskip
			\ex Kraken	\only<5->{\jambox{\alertgreen{Schwa-Tilgung: \textipa{/@/} \ras $\emptyset$},}}
			\only<5->{\jambox{\alertgreen{progressive Ortsassimilation: \textipa{/kn/} \ras \textipa{[kN]}}}}
		\end{xlist}
	
	\end{exe}

\end{itemize}

\end{frame}

%%%%%%%%%%%%%%%%%%%%%%%%%%%%%%%%%%%
\begin{frame}

\begin{itemize}
	\item[5.] Sind die folgenden Segmentfolgen mögliche phonetische Wörter des Standarddeutschen?
	
	\begin{exe}
		\exr{ex:04} \textipa{[p@:kl.\textprimstress Ipl]}
		\exr{ex:05} \textipa{[\textprimstress Na:h.i:ltd]}
	
	\end{exe}
	
	\only<2->{\item \alertgreen{Beispiel (\ref{ex:04}) kann kein phonetisches Wort des Standarddeutschen sein, denn: gespanntes \textipa{[@]}, Verletzung der Sonoritätshierarchie in der Koda oder Onset-Maximierung in der folgenden Silbe \textipa{[kl]}, keine Knacklauteinsetzung in betonter Silbe, Verletzung der Sonoritätshierarchie \textipa{[pl]}}}
	
	\only<3->{\item \alertgreen{Beispiel (\ref{ex:05}) kann kein phonetisches Wort des Standarddeutschen sein, denn: \textipa{[N]} am Wortanfang, \textipa{[h]} wird wortintern nicht realisiert, \textipa{[i:]} muss mit daraffolgendem Konsonanten kurz sein, fehlende Auslautverhärtung \textipa{[d]}, anschließende fehlende Geminatenreduktion \textipa{[tt]} }}
\end{itemize}

\end{frame}
%%%%%%%%%%%%%%%%%%%%%%%%%%%%%%%%%%%

\section{Graphematik}

%%%%%%%%%%%%%%%%%%%%%%%%%%%%%%%%%%
%% UE 2 - 09 Übungen
%%%%%%%%%%%%%%%%%%%%%%%%%%%%%%%%%%

\begin{frame}
\frametitle{Übung: Graphematik -- Lösung}

\begin{itemize}
	\item[6.] Geben Sie Beispiele für die Anwendung der folgenden graphematischen Prinzipien an:
	
	\begin{exe}
		\exr{ex:06}
		
		\begin{xlist}
			\ex \only<1->{Prinzip der Morphemkonstanz} \\
			\alertgreen{\only<2->{\item[-] Silbengelenke, wegen zugehöriger Pluralformen: \zB \textit{Ba\underline{ll}},}}
			\alertgreen{\only<2->{\item[-] Dehnungs-h, wegen zugehöriger Flexionsformen: \zB \textit{de\underline{h}nen, weil du dehnst}}}
	
			\ex \only<1->{Homonymieprinzip} \\
			\alertgreen{\only<3->{\item[-] Differenzierung homophoner Formen: \zB \textit{L\underline{ee}re vs. L\underline{eh}re}}}
	
			\ex \only<1->{Silbisches Prinzip} \\
			\alertgreen{\only<4->{\item[-] Silbengelenk: \zB \textit{Wa\underline{ss}er},}}
			\alertgreen{\only<4->{\item[-] Silbentrennendes h: \zB S\textit{chu\underline{h}e},}}
			\alertgreen{\only<4->{\item[-] Dehnungs-h: \zB \textit{Sa\underline{h}ne},}}
			\alertgreen{\only<4->{\item[-] Gespanntheit: \zB \textit{M\underline{oo}s},}}
		\end{xlist}
	
	\end{exe}
		
\end{itemize}

\end{frame}

%%%%%%%%%%%%%%%%%%%%%%%%%%%%%%%%%%

\begin{frame}
	
\begin{itemize}
	\item[7.] Geben Sie die rein phonographische Schreibung der folgenden Wörter an:	
	
	\begin{exe}
		\exr{ex:07}
	
		\begin{xlist}		
			\ex sprachbegabt \loesung{2}{\ab{schprachbegabt}}
	
			\ex Sträuchersee  \loesung{3}{\ab{schtreucherse}}
		\end{xlist}
	
	\end{exe}
\end{itemize}

\end{frame}
%%%%%%%%%%%%%%%%%%%%%%%%%%%%%%%%%%

\section{Morphologie}

%%%%%%%%%%%%%%%%%%%%%%%%%%%%%%%%%%
%% UE 3 - 09 Übungen
%%%%%%%%%%%%%%%%%%%%%%%%%%%%%%%%%%

\begin{frame}
\frametitle{Übung: Morphologie -- Lösung}

\begin{itemize}
	\item[8.] Welche Wortbildungsprozesse haben hier stattgefunden?
	
	\begin{exe}
		\exr{ex:08}
	\settowidth \jamwidth{\alertgreen{\only<2->{Derivation (Präfigierung) oder Partikelverbbildung}}}
	
		\begin{xlist}
			\ex übersetz(-en)  \jambox{\alertgreen{\only<2->{Derivation (Präfigierung) oder Partikelverbbildung}}}
			\ex bleifrei \jambox{\alertgreen{\only<3->{Rektionskompositum}}}
			\ex Tanz \jambox{\alertgreen{\only<4->{Konversion}}}
			\ex Bearbeitung \jambox{\alertgreen{\only<5->{1. Derivation (Präfigierung),}}}
			\jambox{\alertgreen{\only<5->{2. Derivation (Suffigierung)}}}
		\end{xlist}
	
	\end{exe}

\end{itemize}

\end{frame}

%%%%%%%%%%%%%%%%%%%%%%%%%%%%%%%%%%
\begin{frame}
	
\begin{itemize}
	\item[9.] Geben Sie die Konstituentenstruktur der folgenden Wörter an und bestimmen Sie die Wortbildungstypen an jedem Knoten des Baumes so genau wie möglich.
	
	\begin{exe}
		\exr{ex:09}
		
		\begin{xlist}
			\ex Unbeweisbarkeitsannahmen
			\ex (mit den) Blickbewegungsmessern
		\end{xlist}

	\end{exe}

\end{itemize}

\end{frame}

%%%%%%%%%%%%%%%%%%%%%%%%%%%%%%%%%%

\begin{frame}

\begin{itemize}
	\item Analyse mit Fugenelement \\
	(9a): Unbeweisbarkeitsannahmen

	\scalebox{.68}{
	\alertgreen{
	\begin{forest} MyP edges,
		[N, name=N1
			[N, name=N2
				[N
					[N, name=N3
						[A, name=A1
							[A\MyPup{af} [un-, tier=word]]
							[A, name=A2
								[V, name=V3
									[A\MyPup{af} [be-, tier=word]]
									[V [weis, tier=word]]
								]
								[A\MyPup{af} [-bar, tier=word]]
							]
						]
						[N\MyPup{af} [-keit, tier=word]]
					]
					[FE [-s, tier=word]]
				]
				[N, name=N4
					[V, name=V1
						[V\MyPup{af} [an-, tier=word]]
						[V, name=V2 [nahm/nehm, tier=word]]
					]
					[N [-e, tier=word]]
				]
			]
			[FI [-n, tier=word]]
		]
	\draw[<-, HUgreen] (N1.west)--++(-12.5em,0pt)
	node[anchor=east,align=center]{Flexion (KEIN Wortbildungsprozess)};
	\draw[<-, HUgreen] (N2.west)--++(-13.5em,0pt)
	node[anchor=east,align=center]{Rektionskompositum};
	\draw[<-, HUgreen] (N3.west)--++(-9em,0pt)
	node[anchor=east,align=center]{Derivation};
	\draw[<-, HUgreen] (A1.west)--++(-4.7em,0pt)
	node[anchor=east,align=center]{Derivation};
	\draw[<-, HUgreen] (A2.east)--++(2.5em,0pt)--++(0pt,-22.5ex)
	node[anchor=north,align=center]{Derivation};
	\draw[<-, HUgreen](N4.east)--++(3.5em,0pt)--++(0pt,-43.5ex)
	node[anchor=north,align=center]{Derivation};
	\draw[<-, HUgreen](V1.west)--++(-2.5em,0pt)--++(0pt,-36.7ex)
	node[anchor=north,align=center]{Derivation};
	\draw[<-, HUgreen](V2.west)--++(-2em,0pt)--++(0pt,-32ex)
	node[anchor=north,align=center]{implizite Derivation};
	\draw[<-, HUgreen](V3.west)--++(-2em,0pt)--++(0pt,-15.5ex)
	node[anchor=north,align=center]{Derivation};
	\end{forest} 
	} }
	
\end{itemize}

\end{frame}

%%%%%%%%%%%%%%%%%%%%%%%%%%%%%%%%%%

\begin{frame}

\begin{itemize}
	\item Analyse mit Kompositionsstammform \\
	(9a): Unbeweisbarkeitsannahmen
	
	\scalebox{.75}{
	\alertgreen{
	\begin{forest} MyP edges,
		[N, name=N1
			[N, name=N2
				[N, name=N3
					[A, name=A1
						[A\MyPup{af} [un-, tier=word]]
						[A, name=A2
							[V, name=V3
								[A\MyPup{af} [be-, tier=word]]
								[V [weis, tier=word]]
							]
							[A\MyPup{af} [-bar, tier=word]]
						]
					]
					[N\MyPup{af} [-keit(-s), tier=word]]
				]
				[N, name=N4
					[V, name=V1
						[V\MyPup{af} [an-, tier=word]]
						[V, name=V2 [nahm/nehm, tier=word]]
					]
					[N [-e, tier=word]]
				]
			]
			[FI [-n, tier=word]]
		]
	\draw[<-, HUgreen] (N1.west)--++(-13em,0pt)
	node[anchor=east,align=center]{Flexion (KEIN Wortbildungsprozess)};
	\draw[<-, HUgreen] (N2.west)--++(-13.5em,0pt)
	node[anchor=east,align=center]{Rektionskompositum};
	\draw[<-, HUgreen] (N3.west)--++(-11em,0pt)
	node[anchor=east,align=center]{Derivation};
	\draw[<-, HUgreen] (A1.west)--++(-6.5em,0pt)
	node[anchor=east,align=center]{Derivation};
	\draw[<-, HUgreen] (A2.east)--++(2.5em,0pt)--++(0pt,-22.5ex)
	node[anchor=north,align=center]{Derivation};
	\draw[<-, HUgreen](N4.east)--++(3.5em,0pt)--++(0pt,-36.5ex)
	node[anchor=north,align=center]{Derivation};
	\draw[<-, HUgreen](V1.west)--++(-2.5em,0pt)--++(0pt,-29.7ex)
	node[anchor=north,align=center]{Derivation};
	\draw[<-, HUgreen](V2.west)--++(-2em,0pt)--++(0pt,-25ex)
	node[anchor=north,align=center]{implizite Derivation};
	\draw[<-,HUgreen](V3.west)--++(-2em,0pt)--++(0pt,-15.5ex)
	node[anchor=north,align=center]{Derivation};
	\end{forest} 
	} }

\end{itemize}

\end{frame}

%%%%%%%%%%%%%%%%%%%%%%%%%%%%%%%%%%

\begin{frame}

\begin{itemize}
	\item Analyse mit Fugenelement \\
	(9b): (mit den) Blickbewegungsmessern
			
	\scalebox{.85}{
	\alertgreen{
	\begin{forest} MyP edges,
		[N, name=N1
			[N, name=N2
				[N
					[N, name=N3
						[N [blick, tier=word]]
						[N, name=N4
							[V [beweg, tier=word]]
							[N\MyPup{af} [-ung, tier=word]]
						]
					]
					[FE [-s, tier=word]]
				]
				[N, name=N5
					[V [mess, tier=word]]
					[N\MyPup{af} [-er, tier=word]]
				]
			]
			[FI [-n, tier=word]]
		]
	\draw[<-, HUgreen](N1.west)--++(-11.8em,0pt)
	node[anchor=east,align=center]{Flexion (KEIN Wortbildungsprozess)};
	\draw[<-, HUgreen](N2.west)--++(-14.5em,0pt)
	node[anchor=east,align=center]{Rektionskompositum};
	\draw[<-, HUgreen](N3.west)--++(-7em,0pt)
	node[anchor=east,align=center]{Rektionskompositum};
	\draw[<-, HUgreen](N4.east)--++(2.5em,0pt)--++(0pt,-16.2ex)
	node[anchor=north,align=center]{Derivation};
	\draw[<-, HUgreen](N5.east)--++(2em,0pt)--++(0pt,-30ex)
	node[anchor=north,align=center]{Derivation};
	\end{forest}
	} }

\end{itemize}

\end{frame}	

%%%%%%%%%%%%%%%%%%%%%%%%%%%%%%%%%%

\begin{frame}

\begin{itemize}
	\item Analyse mit Kompositionsstammform \\
	(9b): (mit den) Blickbewegungsmessern
	
	\scalebox{.9}{
	\alertgreen{
	\begin{forest} MyP edges,
		[N, name=N1
			[N, name=N2
				[N, name=N3
					[N [blick, tier=word]]
					[N, name=N4
						[V [beweg, tier=word]]
						[N\MyPup{af} [-ung(-s), tier=word]]
					]
				]
				[N, name=N5
					[V [mess, tier=word]]
					[N\MyPup{af} [-er, tier=word]]
				]
			]
			[FI [-n, tier=word]]
		]
	\draw[<-, HUgreen](N1.west)--++(-11em,0pt)
	node[anchor=east,align=center]{Flexion (KEIN Wortbildungsprozess)};
	\draw[<-, HUgreen](N2.west)--++(-13em,0pt)
	node[anchor=east,align=center]{Rektionskompositum};
	\draw[<-, HUgreen](N3.west)--++(-7.7em,0pt)
	node[anchor=east,align=center]{Rektionskompositum};
	\draw[<-, HUgreen](N4.east)--++(3.5em,0pt)--++(0pt,-16.2ex)
	node[anchor=north,align=center]{Derivation};
	\draw[<-, HUgreen](N5.east)--++(2em,0pt)--++(0pt,-23ex)
	node[anchor=north,align=center]{Derivation};
	\end{forest}
	} }
	
\end{itemize}

\end{frame}
%%%%%%%%%%%%%%%%%%%%%%%%%%%%%%%%%%

\begin{frame}
	
\begin{itemize}
	\item[10.] Geben Sie Beispiele für die folgenden Kompositionsarten an:
	
	\begin{exe}
		\exr{ex:10}
	\settowidth \jamwidth{\alertgreen{Romanleser, Tierkennerin, Konfliktbewältigung, \dots}}
	
		\begin{xlist}
			\ex Determinativkomposition \\ 
			\only<2->{\jambox{\alertgreen{Apfelsaft, Taschenlampe, Hundeleine, \dots}}}
\medskip	
			\ex Rektionskomposition \\ 
			\only<3->{\jambox{\alertgreen{Romanleser, Tierkennerin, Konfliktbewältigung, \dots}}}
\medskip	
			\ex Possessivkomposition \\ 
			\only<4->{\jambox{\alertgreen{Dickkopf, Grünschnabel, Milchgesicht, \dots}}}
\medskip	
			\ex Kopulativkomposition \\ 
			\only<5->{\jambox{\alertgreen{nordost, Berlin-Brandenburg, blaugrau, \dots}}}
		\end{xlist}

	\end{exe}

\end{itemize}

\end{frame}
%%%%%%%%%%%%%%%%%%%%%%%%%%%%%%%%%%

\begin{frame}

\begin{itemize}
	\item[11.] Geben Sie je ein Beispiel für einen Stamm, für eine Wurzel, für eine Basis und für ein unikales Morphem an.
	
	\begin{exe}
		\exr{ex:11}	
	\settowidth \jamwidth{Handyvertrags-laufzeit [hinsichtl. Komposition]X}
	
		\begin{xlist}	
			\ex Stamm: \jambox{\only<2->{\alertgreen{Handyvertrags}-\alertgreen{laufzeit} [hinsichtl. Komposition]}} 
			\jambox{\only<2->{\alertgreen{Handyvertragslaufzeit}-en [hinsichtl. Flexion]}}
	
			\ex Wurzel: \jambox{\only<3->{un-be-\alertgreen{weis}-bar-es}}
	
			\ex Basis: \jambox{\only<4->{un-\alertgreen{beweisbar}}}
	
			\ex unikales Morphem: \jambox{\only<5->{ver-\alertgreen{letz}-en}}
		\end{xlist}
	
	\end{exe}
	
\end{itemize}

\end{frame}
%%%%%%%%%%%%%%%%%%%%%%%%%%%%%%%%%%

\section{Syntax}

%%%%%%%%%%%%%%%%%%%%%%%%%%%%%%%%%%
%% UE 4 - 09 Übungen
%%%%%%%%%%%%%%%%%%%%%%%%%%%%%%%%%%

\begin{frame}
\frametitle{Übung: Syntax -- Lösung}

\begin{itemize}
	\item[12.] Ordnen Sie die folgenden Matrixsätze und ihre Nebensätze in das topologische Feldermodell ein.
	
	\begin{exe}
		\exr{ex:12}
		
		\begin{xlist}
			\ex Petra wirkt müde, obwohl sie nicht viel getanzt hat.
			\ex Wenn ich im Konzert bin, höre ich der Musik zu.
			\ex Die Frau, die hier arbeitet, obwohl die Heizung ausgeschaltet ist, ist leider krank geworden.
			\ex Anke hat gemerkt, dass Maria trotz der Erkältung arbeiten gegangen ist.
		\end{xlist}

	\end{exe}
	
\end{itemize}

\end{frame}
%%%%%%%%%%%%%%%%%%%%%%%%%%%%%%%%%%
\begin{frame}
	
	\begin{table}
		\centering
		\scalebox{.75}{
		\alertgreen{
		\begin{tabular}{p{3.5cm}|l|p{4cm}|p{2cm}|p{3cm}}
			\textbf{VF} & \textbf{LSK} & \textbf{MF} & \textbf{RSK} & \textbf{NF} \\
			\hline
			Petra & wirkt & müde, & & obwohl sie nicht viel getanzt hat.\\
			\hline
			& obwohl & sie nicht viel & getanzt hat. & \\
			\hline
			\hline
			Wenn ich im Konzert bin, & höre & ich der Musik & zu. & \\
			\hline
			& Wenn & ich im Konzert & bin, & \\
			\hline
			\hline
			Die Frau, die hier arbeitet, obwohl die Heizung ausgeschaltet ist, & ist & leider krank & geworden. & \\
			\hline
			die & & hier & arbeitet, & obwohl die Heizung ausgeschaltet ist, \\
			\hline
			& obwohl & die Heizung & ausgeschaltet ist, & \\
			\hline
			\hline
			Anke & hat & & gemerkt, & dass Maria trotz der Erkältung arbeiten gegangen ist. \\
			\hline
			& dass & Maria trotz der Erkältung & arbeiten gegangen ist. & \\
		\end{tabular}
		} }
	\end{table}

\end{frame}
%%%%%%%%%%%%%%%%%%%%%%%%%%%%%%%%%%

\begin{frame}

\begin{itemize}
	
	\item[13.] Testen Sie anhand von jeweils zwei Konstituententests, ob die kursiv gesetzte Wortfolge eine Konstituente des Satzes bildet.
	
	\begin{exe}
		\exr{ex:13}
		
		\begin{xlist}
			\ex Am Ende bekam Jakob das \emph{für Luise vorbereitete} Kostüm.

			\only<2->{
				\begin{itemize}
					\item \alertgreen{Vorfeldtest: *Für Luise vorbereitete bekam am Ende Jakob das Kostüm.}
					\item \alertgreen{Fragetest: Was bekam Jakob am Ende? \textit{*Für Luise vorbereitete.}}
				\end{itemize}
				}
	
			\alertgreen{\only<2->{$\rightarrow$ [für Luise vorbereitete] ist keine Konstituente}}
	
\medskip
	
			\ex Er hatte sich das überlegt, \emph{weil Jakob wieder krank war}.

			\only<3->{
				\begin{itemize}
					\item \alertgreen{Vorfeldtest: Weil Jakob wieder krank war, hatte er sich das überlegt.}
					\item \alertgreen{Fragetest: Warum hatte er sich das überlegt? \textit{Weil Jakob wieder krank war.}}
				\end{itemize}
				}
	
			\alertgreen{\only<3->{$\rightarrow$ [weil Jakob wieder krank war] ist eine Konstituente}}
		\end{xlist}
	
	\end{exe}
		
\end{itemize}

\end{frame}
%%%%%%%%%%%%%%%%%%%%%%%%%%%%%%%%%%

\begin{frame}
	
\begin{itemize}
	\item[14.] Analysieren Sie die folgenden Sätze nach dem X-Bar-Schema.
		
	\begin{exe}
		\exr{ex:14}
			
		\begin{xlist}		
			\ex Maria sieht Peter.
			\ex Weil der nette Nachbar konzentriert gearbeitet hat, hat er erst an dem Abend Peter getroffen.
			\ex Über die Behandlung der zwei Patienten haben bis gestern die Ärzte diskutiert.
			\ex Luise hat gefragt, ob Jakob kommt.
		\end{xlist}
	
	\end{exe}
		
	\end{itemize}

\end{frame}
%%%%%%%%%%%%%%%%%%%%%%%%%%%%%%%%%%

\begin{frame}
	
(14a): Maria sieht Peter.
		
	\centering
	\scalebox{.56}{
			\alertgreen{
					
				\begin{forest}
					sm edges, empty nodes
					[CP
						[DP$_{i}$
							[\MyPxbar{D}
								[\zerobar{D} [$\emptyset$]]
								[NP
									[\MyPxbar{N}
										[\zerobar{N} [Maria]]
									]
								]
							]
						]
						[\MyPxbar{C}
							[\zerobar{C} [sieht$_{ii}$]]
							[IP
								[[t$_{i}$]]
								[\MyPxbar{I}
									[VP
										[\MyPxbar{V}
											[DP
												[\MyPxbar{D}
													[\zerobar{D} [$\emptyset$]]
													[NP
														[\MyPxbar{N}
															[\zerobar{N} [Peter]]
														]
													]
												]
											]
											[\zerobar{V} [t$_{ii}$]]
										]
									]
									[\zerobar{I} [t$_{ii}$]]
								]
							]
						]
					]
				\end{forest}
				}}

\end{frame}

%%%%%%%%%%%%%%%%%%%%%%%%%%%%%%%%%%

\begin{frame}

(14b): Weil der nette Nachbar konzentriert gearbeitet hat, hat er erst an dem Abend Peter getroffen.
			
	\centering
	\scalebox{.45}{
			\alertgreen{
				\begin{forest}
					sm edges, empty nodes
					[CP
						[CP$_{i}$
							[\MyPxbar{C}
								[\zerobar{C} [weil]]
								[IP
									[DP
										[\MyPxbar{D}
											[\zerobar{D} [der]]
											[NP
												[AP
													[\MyPxbar{A}
														[\zerobar{A} [nette]]
													]
												]
												[NP
													[\MyPxbar{N}
														[\zerobar{N} [Nachbar]]
													]
												]
											]
										]
									]
									[\MyPxbar{I}
										[VP
											[AdvP
												[\MyPxbar{Adv}
													[\zerobar{Adv} [konzentriert]]
												]
											]
											[VP
												[\MyPxbar{V}
													[\zerobar{V} [gearbeitet]]
												]
											]
										]
										[\zerobar{I} [hat]]
									]
								]
							]
						]
						[\MyPxbar{C}
							[\zerobar{C} [hat$_{ii}$]]
							[IP
								[DP
									[\MyPxbar{D}
										[\zerobar{D} [er]]
									]
								]
								[\MyPxbar{I}
									[VP
										[PP
											[AdvP
												[\MyPxbar{Adv}
													[\zerobar{Adv} [erst]]
												]
											]
											[PP
												[\MyPxbar{P}
													[\zerobar{P} [an]]
													[DP
														[\MyPxbar{D}
															[\zerobar{D} [dem]]
															[NP
																[\MyPxbar{N}
																	[\zerobar{N} [Abend]]
																]
															]
														]
													]
												]
											]
										]
										[VP
											[[t$_{i}$]]
											[VP
												[\MyPxbar{V}
													[DP
														[\MyPxbar{D}
															[\zerobar{D} [$\emptyset$]]
															[NP
																[\MyPxbar{N}
																	[\zerobar{N} [Peter]]
																]
															]		
														]
													]
												[\zerobar{V} [getroffen]]
												]
											]
										]
									]
									[\zerobar{I} [t$_{ii}$]]
								]
							]
						]
					]			
				\end{forest}
				}}
			
\end{frame}	

%%%%%%%%%%%%%%%%%%%%%%%%%%%%%%%%%%

\begin{frame}

(14c): Über die Behandlung der zwei Patienten haben bis gestern die Ärzte diskutiert.

	\centering
	\scalebox{.47}{
			\alertgreen{
				\begin{forest}
					sm edges, empty nodes
					[CP
						[PP$_{ii}$
							[\MyPxbar{P}
								[\zerobar{P} [über]]
								[DP
									[\MyPxbar{D}
										[\zerobar{D} [die]]
										[NP
											[\MyPxbar{N}
												[\zerobar{N} [Behandlung]]
												[DP
													[\MyPxbar{D}
														[\zerobar{D} [der]]
														[NP
															[AP
																[\MyPxbar{A}
																	[\zerobar{A} [zwei]]
																]
															]
															[NP
																[\MyPxbar{N}
																	[\zerobar{N} [Patienten]]
																]
															]
														]
													]
												]
											]
										]
									]
								]
							]
						]
						[\MyPxbar{C}
							[\zerobar{C} [haben$_{i}$]]
							[IP
								[PP
									[\MyPxbar{P}
										[\zerobar{P} [bis]]
									[AdvP
										[\MyPxbar{Adv}
											[\zerobar{Adv} [gestern]]
										]
									]
									]
								]
								[IP
									[DP
										[\MyPxbar{D}
											[\zerobar{D} [die]]
											[NP
												[\MyPxbar{N}
													[\zerobar{N} [Ärzte]]
												]
											]
										]
									]
									[\MyPxbar{I}
										[VP
											[\MyPxbar{V}
												[[t$_{ii}$]]
												[\zerobar{V} [diskutiert]]
											]
										]
										[\zerobar{I} [t$_{i}$]]
									]
								]
							]
						]
					]
				\end{forest}
				}}
			
\end{frame}

%%%%%%%%%%%%%%%%%%%%%%%%%%%%%%%%%%

\begin{frame}

(14d): Luise hat gefragt, ob Jakob kommt.
			
	\centering
	\scalebox{.63}{
			\alertgreen{
				\begin{forest}
					sm edges, empty nodes
					[CP
						[DP$_{i}$
							[\MyPxbar{D}
								[\zerobar{D} [$\emptyset$]]
								[NP
									[\MyPxbar{N}
										[\zerobar{N} [Luise]]
									]
								]
							]
						]
						[\MyPxbar{C}
							[\zerobar{C} [hat$_{ii}$]]
							[IP
								[IP
									[[t$_{i}$]]
									[\MyPxbar{I}
										[VP
											[\MyPxbar{V}
												[[t$_{iii}$]]
												[\zerobar{V} [gefragt]]
											]
										]
										[\zerobar{I} [t$_{ii}$]]
									]
								]
								[CP$_{iii}$
									[\MyPxbar{C}
										[\zerobar{C} [ob]]
										[IP
											[DP
												[\MyPxbar{D}
													[\zerobar{D} [er]]
												]
											]
											[\MyPxbar{I}
												[VP
													[\MyPxbar{V}
														[\zerobar{V} [t$_{iv}$]]
													]
												]
												[\zerobar{I} [kommt$_{iv}$]]
											]
										]
									]
								]
							]
						]
					]
				\end{forest}
				}}			

\end{frame}

%%%%%%%%%%%%%%%%%%%%%%%%%%%%%%%%%%

\section{Semantik/Pragmatik}

%%%%%%%%%%%%%%%%%%%%%%%%%%%%%%%%%%
%% UE 5 - 09 Übungen
%%%%%%%%%%%%%%%%%%%%%%%%%%%%%%%%%%

\begin{frame}
\frametitle{Übung: Semantik/Pragmatik -- Lösung}

\begin{itemize}
	\item[15.] Geben Sie die Bedeutungsrelationen (so genau wie möglich) zwischen den folgenden Wörtern an.
	
	\begin{exe}
		\exr{ex:15}
	\settowidth \jamwidth{\alertgreen{kontradiktorische Antonymie}}
	
		\begin{xlist}
			\ex satt -- hungrig \only<2->{\jambox{\alertgreen{konträre Antonymie}}}
			\ex erwerben -- kaufen \only<3->{\jambox{\alertgreen{(partielle) Synonymie}}}
			\ex Haare -- Kopf \only<4->{\jambox{\alertgreen{Meronymie}}}
			\ex schuldig -- nicht schuldig \only<5->{\jambox{\alertgreen{kontradiktorische Antonymie}}}
			\ex heute -- Häute \only<6->{\jambox{\alertgreen{Homophonie}}}
			\ex Tiger -- Katze \only<7->{\jambox{\alertgreen{Hyponymie}}}
			\ex fruchtbar -- unfruchtbar \only<8->{\jambox{\alertgreen{kontradiktorische Antonymie}}}
		\end{xlist}
	
	\end{exe}
	
\end{itemize}

\end{frame}

%%%%%%%%%%%%%%%%%%%%%%%%%%%%%%%%%%

\begin{frame}
	
\begin{itemize}
	\item[16.] Illustrieren Sie die Begriffe Satzbedeutung, Äußerungsbedeutung und Sprecherbedeutung mithilfe des folgenden Satzes.
	
	\begin{exe}
		\exr{ex:16} Ich glaube, du gehst jetzt! \\
		\ras Peter zu Klaus am 25. August 2020 um 20:30 Uhr.
	\end{exe}
	
	\item Satzbedeutung: \\ \only<2->{\alertgreen{Der Sprecher des Satzes glaubt (zum Zeitpunkt der Äußerung), dass der Adressat der Äußerung geht.}}
	\item Äußerungsbedeutung: \\ \only<3->{\alertgreen{Klaus glaubt, dass Peter am 25. August 2020 um 20:30 Uhr geht.}}
	\item Sprecherbedeutung: \\ \only<4->{\alertgreen{Peter fordert Klaus bestimmt auf (\zB nach einer Auseinandersetzung) sofort zu gehen.}}
	
\end{itemize}

\end{frame}

%%%%%%%%%%%%%%%%%%%%%%%%%%%%%%%%%%

\begin{frame}

\begin{itemize}
	\item[17.] Geben Sie die Bedeutungsrelationen zwischen den folgenden Sätzen an.
	
	\begin{exe}
		\exr{ex:17}
	\settowidth \jamwidth{\only<2->{\alertgreen{\ras a impliziert b}}}
	
		\begin{xlist}
			\ex Hinter dem Baum steht ein Bär. \jambox{\only<2->{\alertgreen{Implikation}}}
			\ex Hinter dem Baum steht ein Tier. \jambox{\only<2->{\alertgreen{\ras a impliziert b}}}
		\end{xlist}
	
		\exr{ex:18}
		
		\begin{xlist}
			\ex Peter fängt an zu arbeiten. \jambox{\only<3->{\alertgreen{Paraphrase}}}
			\ex Peter nimmt die Arbeit auf.
		\end{xlist}
		
		\exr{ex:19}
		
		\begin{xlist}
			\ex Sandra ist groß. \jambox{\only<4->{\alertgreen{Kontradiktion}}}
			\ex Sandra ist nicht-groß.
		\end{xlist}
		
		\exr{ex:20}
		
		\begin{xlist}
			\ex Ich habe alle Studenten gesehen. \jambox{\only<5->{\alertgreen{Paraphrase}}}
			\ex Ich habe nicht einen Studenten nicht gesehen.
		\end{xlist}
	
		\exr{ex:21}
		
		\begin{xlist}
			\ex Maria geht wandern. \jambox{\only<6->{\alertgreen{Inkompatibilität}}}
			\ex Maria macht eine Kreuzfahrt.
		\end{xlist}
		
		\exr{ex:22}
		
		\begin{xlist}
			\ex Gert ist verletzt. \jambox{\only<7->{\alertgreen{Implikation}}}
			\ex Gert hat ein gebrochenes Bein. \jambox{\only<6->{\alertgreen{\ras b impliziert a}}}
		\end{xlist}
		
	\end{exe}

\end{itemize}
	
\end{frame}

%%%%%%%%%%%%%%%%%%%%%%%%%%%%%%%%%%

\begin{frame}
	
\begin{itemize}
		
	\item[18.] Geben Sie eine Wahrheitswerttabelle für den folgenden aussagenlogischen Ausdruck an und bestimmen Sie, ob es sich dabei um eine tautologische, eine kontradiktorische oder eine kontingente Aussage handelt.
	
	\begin{exe}
		\exr{ex:23} $((p \rightarrow q) \lor q)$
	\end{exe}
		
		\begin{table}
			\centering
			\scalebox{.95}{
				\only<2->{\alertgreen{
					\begin{tabular}{c|c|c|c}
					p & q & $(p \rightarrow q)$ & $((p \rightarrow q) \lor q)$ \\
					\hline
					1 & 1 & 1 & 1 \\
					\hline
					1 & 0 & 0 & 0 \\
					\hline
					0 & 1 & 1 & 1 \\
					\hline
					0 & 0 & 1 & 1 \\
					\end{tabular}
					}}}
		\end{table}

\medskip
	
	\only<2->{\item \alertgreen{Bei dem vorangehenden aussagenlogischen Ausdruck handelt es sich um eine kontingente Aussage.}}
				
\end{itemize}
	
\end{frame}

%%%%%%%%%%%%%%%%%%%%%%%%%%%%%%%%%%

\begin{frame}

\begin{itemize}
	\item[19.] Markieren Sie alle deiktischen und anaphorischen Elemente in den folgenden Sätzen und spezifizieren Sie diese.
		
	\begin{exe}
		\exr{ex:24}
		
		\begin{xlist}
			\ex Sie haben diese Tür nicht geschlossen.
			\ex Gestern war mir das Wetter echt zu kalt!
			\ex Peter wusste, dass er es sich dort gemütlich machen würde.
		\end{xlist}
	
	\end{exe}

\end{itemize}

\end{frame}

%%%%%%%%%%%%%%%%%%%%%%%%%%%%%%%%%%

\begin{frame}

\begin{itemize}
	\item[19.] Markieren Sie alle deiktischen und anaphorischen Elemente in den folgenden Sätzen und spezifizieren Sie diese.
	
	\begin{exe}
		\exr{ex:24}
		
		\begin{xlist}
			\ex \alertgreen{Sie} haben \alertgreen{diese} Tür nicht geschlossen.
			\ex \alertgreen{Gestern} war \alertgreen{mir} das Wetter echt zu kalt!
			\ex Peter wusste, dass \alertblue{er} es \alertblue{sich} \alertgreen{dort} gemütlich machen würde. 
			%	\ex \gqq{Ich bin sehr glücklich, mich wieder für das WTA-Finale qualifiziert zu haben. Ich freue mich darauf, dort anzutreten und gegen die Besten der Welt zu spielen}, sagte die 27-Jährige, die im vergangenen Jahr nur als Ersatzspielerin mitfahren durfte.
		\end{xlist}
	
	\end{exe}

\end{itemize}	
	
	\begin{minipage}[t]{0.35\textwidth}

	\begin{itemize}
		\item \alertgreen{Deiktische Elemente:}
		\only<2->{
		
			\begin{itemize}
				 \item \only<2->{\alertgreen{Sie: Sozialdeixis}}
				 \item \only<2->{\alertgreen{diese: Objektdeixis}}
				 \item \only<2->{\alertgreen{gestern: Temporaldeixis}}
				 \item \only<2->{\alertgreen{mir: Personaldeixis}}
				 \item \only<2->{\alertgreen{dort: Lokaldeixis}} 
			\end{itemize}		
			}

	\end{itemize}
	
	\end{minipage}
	\begin{minipage}[t]{0.60\textwidth}
	
	\begin{itemize}
		\item \alertblue{Anaphorische Elemente:}
		\only<2->{
		
			\begin{itemize}
				\item \only<2->{\alertblue{er: anaphorischer Ausdruck; Antezedens: \textit{Peter}}}
				\item \only<2->{\alertblue{sich: anaphorischer Ausdruck; Antezedens: \textit{er}}}
			\end{itemize}
			}
		
	\end{itemize}

	\end{minipage}


\end{frame}

%%%%%%%%%%%%%%%%%%%%%%%%%%%%%%%%%%

\begin{frame}
	
\begin{itemize}
	\item[20.] Bestimmen Sie die Art von Folgerung (Implikation, Präsupposition, Implikatur), die zwischen dem ersten und den folgenden Sätzen besteht:
	
	\begin{exe}
		\exr{ex:25} Sogar Peter hat zwei Kinder.
	\settowidth \jamwidth{\only<2->{\alertgreen{konversationalle Implikatur}}}
		
		\begin{xlist}
			\ex Peter hat nicht mehr als zwei Kinder. \jambox{\only<2->{\alertgreen{konversationalle Implikatur}}}
			\ex Es gibt ein Individuum namens Peter. \jambox{\only<3->{\alertgreen{Präsupposition}}}
			\ex Peter ist Vater. \jambox{\only<4->{\alertgreen{semantische Implikation}}}
			\ex Peter hat vier Kinder. \jambox{\only<5->{\alertgreen{keine Folgerung}}}
			\ex Überraschenderweise hat Peter Kinder. \jambox{\only<6->{\alertgreen{konventionelle Implikatur}}}
		\end{xlist} 
		
	\end{exe}

\end{itemize}
	
\end{frame}

%%%%%%%%%%%%%%%%%%%%%%%%%%%%%%%%%%

\begin{frame}

\begin{itemize}
	\item[21.] Bestimmen Sie jeweils eine semantische Implikation aus den folgenden Sätzen:
	
	\begin{exe}
		\exr{ex:26}
		
		\begin{xlist}
			\ex In einem Schuhkarton gibt es Platz für zwei Schuhe.
			\ex Saskia hat eine Schwedin geheiratet.
		\end{xlist}

	\end{exe}
	
\end{itemize}
	
	\only<2->{\alertgreen{(26a): $\vDash$ In einem Schuhkarton gibt es Platz für einen Schuh.}} ~\\
\medskip
	\only<3->{\alertgreen{(26b): $\vDash$ Saskia hat eine Nordeuropäerin geheiratet.}}
	

\end{frame}

%%%%%%%%%%%%%%%%%%%%%%%%%%%%%%%%%%

\begin{frame}
	
\begin{itemize}
		
	\item[22.] Bestimmen Sie jeweils eine Präsupposition aus den folgenden Sätzen:
	
	\begin{exe}
		\exr{ex:27}
		
		\begin{xlist}
			\ex Ich freue mich darüber, dass wir die Klausur bestanden haben.
			\ex Auch Maria ist schwanger.
			\ex Alle Geiseln wurden gerettet.
			\ex Sie mögen immer noch Syntax.
		\end{xlist}

	\end{exe}
		
\end{itemize}

	\only<2->{\alertgreen{(27a): \prspp Wir haben die Klausur bestanden.}} ~\\
\medskip
	\only<3->{\alertgreen{(27b): \prspp Mindestens eine weitere Entität neben Maria ist schwanger.}} ~\\
\medskip
	\only<4->{\alertgreen{(27c): \prspp Die Geiseln waren in Gefahr.}} ~\\
\medskip
	\only<5->{\alertgreen{(27d): \prspp Sie mochten bisher Syntax.}}
	
\end{frame}

%%%%%%%%%%%%%%%%%%%%%%%%%%%%%%%%%%

\begin{frame}
	
\begin{itemize}
	\item[23.] Geben Sie an, ob eine Maxime (scheinbar) verletzt oder befolgt wurde und um welche es sich handelt, um die angegebene Implikatur zu erhalten.
	
	\begin{exe}
		\exr{ex:28} Wir haben einige Personen entlassen.\\
		$+>$ Es wurden nicht alle entlassen.
	\end{exe}
	
	\begin{exe}
		\exr{ex:29} A: Wie war das Bewerbungsgespräch?\\
		B: Das Wetter ist ja super heute!\\
		$+>$ Es war furchtbar!
	\end{exe}
	
\end{itemize}

	\only<2->{\alertgreen{(\ref{ex:28}): Befolgung der Quantitätsmaxime}} ~\\
\medskip
	\only<3->{\alertgreen{(\ref{ex:29}): (scheinbare) Verletzung der Relevanzmaxime}}

\end{frame}

%%%%%%%%%%%%%%%%%%%%%%%%%%%%%%%%%%


%%%%%%%%%%%%%%%%%%%%%%%%%%%%%%%%%%%
%%%%%%%%%%%%%%%%%%%%%%%%%%%%%%%%%%%
%\section{Hausaufgaben}


%% -*- coding:utf-8 -*-

%%%%%%%%%%%%%%%%%%%%%%%%%%%%%%%%%%%%%%%%%%%%%%%%%%%%%%%%%


\def\insertsectionhead{\refname}
\def\insertsubsectionhead{}

\huberlinjustbarfootline


\ifpdf
\else
\ifxetex
\else
\let\url=\burl
\fi
\fi
\begin{multicols}{2}
{\tiny
%\beamertemplatearticlebibitems

\bibliography{../gkbib,../bib-abbr,../biblio}
\bibliographystyle{../unified}
}
\end{multicols}





%% \section{Literatur}
%% \begin{frame}[allowframebreaks]
%% \frametitle{Literatur}
%% 	\footnotesize

%% \bibliographystyle{unified}

%% 	%German
%% %	\bibliographystyle{deChicagoMyP}

%% %	%English
%% %	\bibliographystyle{chicago} 

%% 	\bibliography{gkbib,bib-abbr,biblio}
	
%% \end{frame}



\end{document}